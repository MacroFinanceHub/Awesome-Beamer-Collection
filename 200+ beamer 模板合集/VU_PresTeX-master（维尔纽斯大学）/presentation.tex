%===============================================================================
% Purpose:   Template for \LaTeX beamer presentations at Vilnius University
% Created: Sep 15 2013, Version 1.0 from Mar 17 2013
% Autor:  Linus Dietz (linus.dietz@uni-bamberg.de), fork of Marcel Grossmann (marcel.grossmann@uni-bamberg.de)
%===============================================================================


%===============================================================================
% Run pdflatex and bibtex to compile. Use the Makefile for all in once compilation.
%	Configuration in texmaker:  pdflatex -synctex=1 -interaction=nonstopmode %.tex | bibtex % | pdflatex -synctex=1 -interaction=nonstopmode %.tex | pdflatex -synctex=1 -interaction=nonstopmode %.tex
% Edit Information in config/metainfo 
% Choose the language with the following \lang command
% Options: {english || lithuanian}
\newcommand{\lang}{lithuanian}
%===============================================================================


\documentclass[11pt,\lang ,%
%draft,
%handout,
compress%
]{beamer}

\usepackage{ifthen}

\newcommand{\univustring}{\ifthenelse{\equal{\lang}{lithuanian}}{Vilniaus Universitetas}{Vilnius University}}

% Differential operators
\newcommand{\diff}{\ensuremath{\mathrm{d}}}
\newcommand{\subsdiff}{\ensuremath{\mathrm{D}}}
\newcommand{\vardiff}{\ensuremath{\mathrm{\delta}}}

% Vectors and matrices
\renewcommand{\vec}[1]{\ensuremath{\boldsymbol{#1}}}   % vector in bold
\newcommand{\mat}[1]{\ensuremath{\mathsf{#1}}}	       % matrix in serif font

% Degree Celcius
\newcommand{\degC}{\ensuremath{^\circ \mathrm{C}}}



\def\signed #1{{\leavevmode\unskip\nobreak\hfil\penalty50\hskip2em
  \hbox{}\nobreak\hfil(#1)%
  \parfillskip=0pt \finalhyphendemerits=0 \endgraf}}

\newsavebox\mybox
\newenvironment{aquote}[1]
  {\savebox\mybox{#1}\begin{fancyquotes}}
  {\signed{\usebox\mybox}\end{fancyquotes}}


\hyphenation{op-tical net-works semi-conduc-tor}

\setbeamertemplate{caption}[numbered]
%\numberwithin{figure}{section}
\begin{document}

\frame{\titlepage}

\AtBeginSection[]
{
  \frame<handout:0>
  {
    \frametitle{Outline}
    \tableofcontents[currentsection,hideallsubsections]
  }
}

\AtBeginSubsection[]
{
  \frame<handout:0>
  {
    \frametitle{Outline}
    \tableofcontents[sectionstyle=show/hide,subsectionstyle=show/shaded/hide,subsubsectionstyle=hide]
  }
}

\AtBeginSubsubsection[]
{
  \frame<handout:0>
  {
    \frametitle{Outline}
    \tableofcontents[sectionstyle=show/hide,subsectionstyle=show/shaded/hide,subsubsectionstyle=show/shaded/hide]
  }
}

\newcommand<>{\highlighton}[1]{%
  \alt#2{\structure{#1}}{{#1}}
}

\newcommand{\icon}[1]{\pgfimage[height=1em]{#1}}

\section*{}
\begin{frame}{Content}
\tableofcontents
\end{frame}

%%%%%%%%%%%%%%%%%%%%%%%%%%%%%%%%%%%%%%%%%
%%%%%%%%%% Content starts here %%%%%%%%%%
%%%%%%%%%%%%%%%%%%%%%%%%%%%%%%%%%%%%%%%%%

\section{Intro}
\begin{frame}
\frametitle{Title}
\framesubtitle{Subtitle}
This is a template for presentations in \LaTeX ~beamer. 

A slide can have a title and a subtitle.
\end{frame}

\section{Basic Commands}

\begin{frame}
\frametitle{Basic Commands}
\framesubtitle{\texttt{enumerate} \& \texttt{itemize}}
If you want a 2 column layout, use the \texttt{multicols} environment:
\begin{multicols}{2}
\begin{enumerate}
\item First
\item Second
\item Third
\end{enumerate}
\begin{itemize}
\item This
\item and
\item that
\end{itemize}
\end{multicols}
\end{frame}

\begin{frame}
\frametitle{Basic Commands}
\framesubtitle{Images}
\image{.4}{logo.png}{Vilnius University Logo}{img:logo}

Images can be included as usual -- or with the \texttt{\textbackslash image} command.
\end{frame}

\section{Blocks}

\begin{frame}
\frametitle{Blocks}

\begin{block}{Blocktitle}
This is a normal block.
\end{block}

\begin{alertblock}{Alert}
This is an alert block.
\end{alertblock}


\begin{exampleblock}{Example}
   This is an example block.
\end{exampleblock}

\end{frame}

\section{References}

\begin{frame}
\frametitle{References}
References can be used with the \texttt{BibTeX} commands  \cite{Knuth.1986}. The list of references  will be shown at the end of the presentation with the preferred style.
\end{frame}

%%%%%%%%%%%%%%%%%%%%%%%%%%%%%%%%%%%%%%%%%
%%%%%%%%%%       References      %%%%%%%%
%%%%%%%%%%%%%%%%%%%%%%%%%%%%%%%%%%%%%%%%%

\section*{}
\begin{frame}[allowframebreaks]{References}
\def\newblock{\hskip .11em plus .33em minus .07em}
\scriptsize
\bibliographystyle{alpha}
\bibliography{literature/bib}
\normalsize
\end{frame}

%% Last frame
\frame{
  \vspace{2cm}
 \ifthenelse{\equal{\lang}{lithuanian}}{{\huge Klausimai?}}{\huge Questions ?}

  \vspace{20mm}
  \nocite*
  
  \begin{flushright}  
    Linus Dietz
    
    \structure{\footnotesize{\href{mailto:linus.dietz@uni-bamberg.de}{linus.dietz@uni-bamberg.de}}}
  \end{flushright}
}


\end{document}
