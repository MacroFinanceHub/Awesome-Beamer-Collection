\documentclass[nonav,sleutel]{beamer}
\usepackage[utf8]{inputenc}
\usepackage[T1]{fontenc}
\title{There Is No Largest Prime Number}
\date[ISPN '80]{27th International Symposium of Prime Numbers}
\author[Euclid]{Euclid of Alexandria \texttt{euclid@alexandria.edu}}

\usetheme{kuleuvenstijl}

\begin{document}

\begin{frame}
\titlepage
\end{frame}


\begin{frame} 
\frametitle{There Is No Largest Prime Number} 
\framesubtitle{The proof uses \textit{reductio ad absurdum}.} 
\begin{theorem}
There is no largest prime number. \end{theorem} 
\begin{enumerate} 
\item<1-| alert@1> Suppose $p$ were the largest prime number. 
\item<2-> Let $q$ be the product of the first $p$ numbers. 
\item<3-> Then $q+1$ is not divisible by any of them. 
\item<1-> But $q + 1$ is greater than $1$, thus divisible by some prime
number not in the first $p$ numbers.
\end{enumerate}
\end{frame}

\begin{frame}{A longer title}
\begin{itemize}
\item one
\item two
\end{itemize}

One can prove that
\[
	1 = 1
\]
\end{frame}

\begin{frame}{Blocks}
\begin{block}{Block title}
Block body.
\end{block}
\begin{example}
For clarity:
\begin{itemize}
	\item[$\rightarrow$] first bullet point \ldots
	\item[$\rightarrow$] second bullet  point \ldots
\end{itemize}
\end{example}
\end{frame}

\end{document}