% -*- LaTeX -*-
\usepackage{xcolor}
\usepackage{listings}
\title{Ein Beamer-Stil für die TU Dresden}
\author{Tobias Schlemmer}
\begin{document}
\maketitle
\frame{\tableofcontents}
\section{Hinweis auf die Dokumentation}
\begin{frame}{Dokumentation}
  Die Dokumentation befindet sich in der Datei
  \url{beamer-org-mode-demo.pdf}.
\end{frame}
\section{Einführung}
\begin{frame}

Für das Corporate Design der TU Dresden gibt es seit geraumer Zeit eine Präsentationsklasse zum Erstellen von \LaTeX{}"=Dokumenten mit Hilfe von Beamer. Leider gab es bei dieser Klasse immer wieder Probleme im Zusammenspiel mit anderen Paketen. Dies führte dazu, dass sich nach und nach der vorliegende Beamerstil entwickelt hat. Er versucht einige der Probleme der Klasse auszugleichen:
\end{frame}
\begin{frame}
  \begin{block}{Nachteile von tudbeamer.cls}
    \begin{itemize}
    \item Keine Benutzung von beamerarticle.sty möglich
    \item Lädt veraltetes Paket ngerman.sty und provoziert Inkompatibilitäten
    \item Benutzt Tabellen für das Layout und beißt sich mit xcolor.sty und colortbl.sty
    \item Fehlerhafter Zeilenabstand zwischen vorletzter und letzter Titelzeile
    \item Monolithisch: Arbeit an Fonts und Layout doppelt sich – schwer zu warten.
    \item Falscher unterer und rechter Rand
    \end{itemize}
  \end{block}
  Wir wollen hier aber nicht die Arbeit anderer schlecht reden, sondern in die Benutzung einführen.
\end{frame}

\section{Benutzung}

\begin{frame}[allowframebreaks]{Einbindung}
  Die Klasse kann einfach eingebunden werden:
  \begin{block}{}
    \texttt{\textbackslash usetheme\{tud\}}
  \end{block}
  Und schon erscheint die Präsentation im Corporate Design der TU Dresden. 
  Dabei werden die Schriftarten in folgender Reihenfolge gesucht:
  \begin{enumerate}
  \item Schriften des \texttt{tudscr}-Paketes
  \item \texttt{tudscrold}-Schriften
  \item Schriften der alten TUD-\LaTeX-Klassen
  \end{enumerate}

  Das CD"=Handbuch enthält zu wenig brauchbare Vorgaben, um ein ganz einheitliches Layout vorzugeben. Das ist sicherlich teilweise beabsichtigt. Insofern gibt es auch verschiedene Optionen, mit denen das Layout angepasst werden kann. Die meisten wurden von tudbeamer.cls geerbt. Die beiden Optionen „nogerman“ und „german“ entfallen. Verwenden Sie stattdessen bitte
  \begin{block}{}
    \textbackslash usepackage[ngerman]\{babel\}
  \end{block}
  für Ihren Deutschen Text. Das Paket (n)german.sty ist veraltet und zu einigen Paketen inkompatibel.
\end{frame}

\begin{frame}[allowframebreaks]{Optionen}
  \begin{description}
    \item[heavyfont] Stärkere Schriften
    \item[nodin] Lade keine {\dinfamily DIN bold}
    \item[beamerfont] Keine TU"=Schriften
    \item[serifmath] Benutze die vorgegebene Serifenschrift für mathematische Formeln
    \item[noheader] Keine Kopfzeile mit Logo (außer Titelseite)
    \item[smallrightmargin] Benutze verringerten rechten Rand von tudbeamer.cls
    \item[pagenum] Seitennummern in der Fußzeile
    \item[nosectionnum] Keine Abschnittsnummern in Folienüberschriften
    \item[navbar] Navigationszeile
    \item[ddc] Logo von Dresden Concept als Zweitlogo auf der
      Titelseite (benötigt Logo"=Datei von tudbeamer.cls). Diese
      Option ist für Präsentationen im Zusammenhang mit Dresden
      Concept vorbehalten.
    \item[ddcfooter] Logo von Dresden Concept in der Fußzeile der
      Titelseite (Voreinstellung, benötigt Logo"=Datei von
      tudbeamer.cls). Diese Option ist für alle Präsentationen der TUD
      gedacht, die nicht im Rahmen von Dresden Concept abgehalten
      werden.
    \item[noddc] Es wird kein Logo von Dresden Concept angezeigt
  \end{description}
\end{frame}



\begin{frame}{Schlussbemerkungen}
  \begin{itemize}
  \item Dies Titelseite erzeugen Sie mit \textbackslash maketitle.
  \item Alle Einstellmöglichkeiten werden in den Dateien beamer*.sty
    definiert.\footnote{Das ist die Ausrede derer, die zu faul sind,
      eine ordentliche Dokumentation zu schreiben, oder die aus anderen Gründen keine Zeit haben.}
  \item Alle Fragen, die dann noch bleiben können gerne auf \url{http://github/tud-cd/tud-cd} als neues „Issue“ eröffnet und diskutiert werden.
  \item Darüber hinaus wäre sicherlich eine ausführliche Dokumentation der einzelnen Einstellungen sinnvoll. Wer dort helfen will, kann gern auch gern ein „Issue“ auf \url{http://github/tud-cd/tud-cd} eröffnen.
  \end{itemize}
\end{frame}


\begin{frame}
  \begin{block}{}
    \centerline{\huge\textbf{Viel Spaß!}}
  \end{block}
  \vfill P.S.: Die Beigelegte Präsentation ist ein Beispiel für die
  Verwendung der Klasse, aber als Präsentation völlig
  ungeeignet. Tipps für Ihre Präsentation können sie u.\,a.\ der Datei
  beameruserguide.pdf ihrer \TeX"=Installation entnehmen.
\end{frame}
\end{document}

%%% Local Variables: 
%%% mode: latex
%%% TeX-master: t
%%% End: 
