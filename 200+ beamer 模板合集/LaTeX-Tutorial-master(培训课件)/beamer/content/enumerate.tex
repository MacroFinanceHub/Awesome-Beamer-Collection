\section{Enumerations}
\begin{frame}
\frametitle{Enumerations}

\begin{block}{New commands in this section}
\begin{itemize}
\item \color{unibablueI}\textbackslash begin\color{black}\{itemize\} \ldots \color{unibablueI}\textbackslash end\color{black}\{itemize\} 
\item \color{unibablueI}\textbackslash begin\color{black}\{enumerate\} \ldots \color{unibablueI}\textbackslash end\color{black}\{enumerate\} 
\item \color{nounibaredI}\textbackslash item\color{black}
\end{itemize}
\end{block}
\end{frame}

\begin{frame}
\frametitle{Enumerations}
\framesubtitle{Bulleted List}

\begin{columns}
\begin{column}{.5\textwidth}
\begin{ttfamily}
\color{nounibaredI}\color{nounibaredI}\textbackslash documentclass\color{black}\{article\} \\
\color{nounibaredI}\color{unibablueI}\textbackslash\color{unibablueI}begin\color{black}\color{black}\{document\} \\
\color{nounibaredI}\color{unibablueI}\textbackslash\color{unibablueI}begin\color{black}\color{black}\{itemize\} \\
\color{nounibaredI}\color{nounibaredI}\textbackslash item \color{black} first bullet item \\
\color{nounibaredI}\color{nounibaredI}\textbackslash item \color{black} second bullet item \\
\color{nounibaredI}\color{nounibaredI}\textbackslash item \color{black} third bullet item \\
\color{nounibaredI}\color{nounibaredI}\textbackslash item \color{black} last bullet item \\
\color{nounibaredI}\color{unibablueI}\textbackslash\color{unibablueI}end\color{black}\color{black}\{itemize\} \\
\color{nounibaredI}\color{unibablueI}\textbackslash\color{unibablueI}end\color{black}\color{black}\{document\} \\
\end{ttfamily}
\end{column}
\begin{column}{.5\textwidth}
\begin{itemize}
\item first bullet item
\item second bullet item
\item third bullet item
\item last  bullet item
\end{itemize}
\end{column}
\end{columns}
\bigskip

The particular bullet items are indicated in the „itemize“-environment with the command \begin{ttfamily}\color{nounibaredI}\textbackslash item\color{black}\end{ttfamily}.
\end{frame}

\begin{frame}
\frametitle{Enumerations}
\framesubtitle{Interlacing}

\begin{columns}
\begin{column}{.5\textwidth}
\begin{ttfamily}
\color{nounibaredI}\color{nounibaredI}\textbackslash documentclass\color{black}\{article\} \\
\color{nounibaredI}\color{unibablueI}\textbackslash\color{unibablueI}begin\color{black}\color{black}\{document\} \\
\color{nounibaredI}\color{unibablueI}\textbackslash\color{unibablueI}begin\color{black}\color{black}\{itemize\} \\
\color{nounibaredI}\color{nounibaredI}\textbackslash item \color{black} first bullet item \\
\color{nounibaredI}\color{nounibaredI}\textbackslash item \color{black} second bullet item \\
\color{nounibaredI}\color{unibablueI}\textbackslash\color{unibablueI}begin\color{black}\color{black}\{itemize\} \\
\color{nounibaredI}\color{nounibaredI}\textbackslash item \color{black} first subitem \\
\color{nounibaredI}\color{nounibaredI}\textbackslash item \color{black} second subitem \\
\color{nounibaredI}\color{unibablueI}\textbackslash\color{unibablueI}end\color{black}\color{black}\{itemize\} \\
\color{nounibaredI}\color{nounibaredI}\textbackslash item \color{black} third bullet item \\
\color{nounibaredI}\color{nounibaredI}\textbackslash item \color{black} last bullet item \\
\color{nounibaredI}\color{unibablueI}\textbackslash\color{unibablueI}end\color{black}\color{black}\{itemize\} \\
\color{nounibaredI}\color{unibablueI}\textbackslash\color{unibablueI}end\color{black}\color{black}\{document\} \\
\end{ttfamily}
\end{column}
\begin{column}{.5\textwidth}
\begin{itemize}
\item first bullet item
\item second bullet item
\begin{itemize}
\item first subitem
\item second subitem
\end{itemize}
\item third bullet item
\item last bullet item
\end{itemize}
\end{column}
\end{columns}
\bigskip
In this way, subitems can be interlaced in maximal four layers.
\end{frame}


\begin{frame}
\frametitle{Enumerations}
\framesubtitle{Numerations}

\begin{columns}
\begin{column}{.5\textwidth}
\begin{ttfamily}
\color{nounibaredI}\color{nounibaredI}\textbackslash documentclass\color{black}\{article\} \\
\color{nounibaredI}\color{unibablueI}\textbackslash\color{unibablueI}begin\color{black}\color{black}\{document\} \\
\color{nounibaredI}\color{unibablueI}\textbackslash\color{unibablueI}begin\color{black}\color{black}\{enumerate\} \\
\color{nounibaredI}\color{nounibaredI}\textbackslash item \color{black} first bullet item \\
\color{nounibaredI}\color{unibablueI}\textbackslash\color{unibablueI}begin\color{black}\color{black}\{enumerate\} \\
\color{nounibaredI}\color{nounibaredI}\textbackslash item \color{black} first subitem \\
\color{nounibaredI}\color{nounibaredI}\textbackslash item \color{black} second subitem \\
\color{nounibaredI}\color{unibablueI}\textbackslash\color{unibablueI}end\color{black}\color{black}\{enumerate\} \\
\color{nounibaredI}\color{nounibaredI}\textbackslash item \color{black} second bullet item \\
\color{nounibaredI}\color{nounibaredI}\textbackslash item \color{black} and so forth \\
\color{nounibaredI}\color{unibablueI}\textbackslash\color{unibablueI}end\color{black}\color{black}\{enumerate\} \\
\color{nounibaredI}\color{unibablueI}\textbackslash\color{unibablueI}end\color{black}\color{black}\{document\} \\
\end{ttfamily}
\end{column}
\begin{column}{.5\textwidth}
\begin{enumerate}
\item first bullet item
\begin{enumerate}
\item first subitem
\item second subitem
\end{enumerate}
\item second bullet item
\item and so forth
\end{enumerate}
\end{column}
\end{columns}
\bigskip
Again, the particular bullet items are indicated with the command \begin{ttfamily}\color{nounibaredI}\textbackslash item\color{black}\end{ttfamily}. 
Again there are four layers for interlacing.
\end{frame}

\begin{frame}
\frametitle{Mixed Enumerations?}
\framesubtitle{Of Course!}

\begin{columns}
\begin{column}{.5\textwidth}
\begin{ttfamily}
\color{nounibaredI}\color{unibablueI}\textbackslash\color{unibablueI}begin\color{black}\color{black}\{enumerate\} \\
\color{nounibaredI}\color{nounibaredI}\textbackslash item \color{black} first bullet item \\
\color{nounibaredI}\color{nounibaredI}\textbackslash item \color{black} \color{nounibaredI}\color{unibablueI}\textbackslash\color{unibablueI}begin\color{black}\color{black}\{itemize\} \\
\color{nounibaredI}\color{nounibaredI}\textbackslash item \color{black} first subitem \\
\color{nounibaredI}\color{nounibaredI}\textbackslash item \color{black} second subitem \\
\color{nounibaredI}\color{unibablueI}\textbackslash\color{unibablueI}end\color{black}\color{black}\{itemize\} \\
\color{nounibaredI}\color{nounibaredI}\textbackslash item \color{black} third bullet item \\
\color{nounibaredI}\color{unibablueI}\textbackslash\color{unibablueI}begin\color{black}\color{black}\{enumerate\} \\
\color{nounibaredI}\color{nounibaredI}\textbackslash item \color{black} I matter, too! \\
\color{nounibaredI}\color{unibablueI}\textbackslash\color{unibablueI}end\color{black}\color{black}\{enumerate\} \\
\color{nounibaredI}\color{nounibaredI}\textbackslash item \color{black} and so forth \\
\color{nounibaredI}\color{unibablueI}\textbackslash\color{unibablueI}end\color{black}\color{black}\{enumerate\} \\
\end{ttfamily}
\end{column}
\begin{column}{.5\textwidth}
\begin{enumerate}
\item first bullet item
\item \begin{itemize}
\item first subitem
\item second subitem
\end{itemize}
\item third bullet item
\begin{enumerate}
\item I matter, too!
\end{enumerate}
\item and so forth
\end{enumerate}
\end{column}
\end{columns}
\bigskip
You can customize the symblos with \color{nounibaredI}\textbackslash item\color{nounibagreenI}[]\color{black}. 
\end{frame}