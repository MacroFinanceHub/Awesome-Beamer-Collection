\section{Intro} 

\subsection{Scope, Sense and Purpose}
\begin{frame}[t]
\frametitle{Introduction}
\framesubtitle{Sense -- Nonsense -- Madness}
\bigskip
\bigskip
\bigskip

\begin{columns}[t]
\begin{column}{.3\textwidth}
\textbf{Useful}\\[3mm]
\begin{itemize}
\item Articles
\item Books
\item Scientific papers
\item Applications
\end{itemize}
\end{column}
\begin{column}{.30\textwidth}
\textbf{Nonsense}\\[3mm]
\begin{itemize}
\item Private mail
\item Invitations to your birthday party
\item Drink Menu
\end{itemize}
\end{column}
\begin{column}{.3\textwidth}
\textbf{Madness}\\[3mm]
\begin{itemize}
\item Shopping list
\item Brainstorming
\item \ldots 
\end{itemize}
\end{column}
\end{columns}
\end{frame}

%-------------------------------------------------------------------------------

\begin{frame} 
\frametitle{From Code To Document}
\framesubtitle{No WYSIWYG} 
\begin{columns}
\begin{column}{.7\textwidth}
\image{\textwidth}{image/worddoc.jpg}{\textbf{W}hat \textbf{Y}ou \textbf{S}ee \textbf{I}s \textbf{W}hat \textbf{Y}ou
\textbf{G}et}{img:worddoc}
\end{column}
\begin{column}{.3\textwidth}
\image{\textwidth}{image/codescreen.png}{\textbf{W}hat \textbf{W}ill \textbf{I} \textbf{G}et?}{img:codescreen}
%% Compile Animation
\end{column}
\end{columns}
\end{frame}

%-------------------------------------------------------------------------------

\begin{frame}
\frametitle{Introduction}
\framesubtitle{Approach}
\begin{columns}[onlytextwidth]
\begin{column}{0.40\textwidth}
\image{.8\textwidth}{image/codescreen.png}{Text File with \LaTeX ~-Code}{img:code}
\end{column}
\begin{column}{0.25\textwidth}
\image{.8\textwidth}{image/miktex.jpg}{Compiler (z.B. MikTeX)}{img:miktex}
\end{column}
\begin{column}{0.25\textwidth}
\image{.6\textwidth}{image/pdflogo.png}{good-looking, legible and printable document}{img:pdf}
\end{column}
\end{columns}
\end{frame}


%-------------------------------------------------------------------------------

\subsection{Advantages \& Disadvantages}
\begin{frame}
\frametitle{Introduction}
\framesubtitle{Advantages \& Disadvantages}
Advantages
\begin{itemize}
\item  dynamic directories and references
\item  automated layouts
\item  simple distributed work
\end{itemize}
Disadvantages
\begin{itemize}
\item  What do I get in the end?
\item  many, partly complex commands
\end{itemize}
\end{frame}

%-------------------------------------------------------------------------------

\subsection{\LaTeX --Compiler}
\begin{frame}
\frametitle{Introduction}
\framesubtitle{\LaTeX - Compiler}
\begin{columns}[t]
\begin{column}{.4\textwidth}
\textbf{Software for Windows:}\\
\begin{itemize}
  \item MikTex (http://www.miktex.org)\\
   2 alternatives: Basic or  Complete
  \item ProTeXt (http://www.tug.org/protext)\\
contains MikTex, TeXnicCenter and Ghostscript – simple installation\\
\end{itemize}
\end{column}
\begin{column}{.6\textwidth}
\textbf{Software for *nix:}
\begin{itemize}
  \item TeXLive\\
Packages for Ubuntu: {\ttfamily texlive-full} is the full meta-package with all necessary packages. Also contains the following:
\begin{itemize}
  \item {\ttfamily texlive-base
  \item texlive-lang-german}
\end{itemize}
Installation: {\ttfamily sudo apt-get install texlive-full}
\item MacOS: MacTeX (http://www.tug.org/mactex/2009)\\
\end{itemize}
\end{column}
\end{columns}
\end{frame}

%-------------------------------------------------------------------------------

\subsection{Free Editors}

\subsubsection{*nix}
\begin{frame}
\frametitle{Introduction}
\framesubtitle{Free Editors -- Linus \& MacOS }
\begin{itemize}
 \item Kile\footnote{http://kile.sourceforge.net/}\\KDE-program, also usable under Gnome\slash Unity etc.
 Installation for Debian systemes with {\ttfamily sudo apt-get
 install kile}.
  \item Vim \LaTeX -suite (Plugin)\footnote{http://vim-latex.sourceforge.net/}\\
  Dream for Vim users.
  \item TexShop (MacOS)\footnote{http://pages.uoregon.edu/koch/texshop/}
\end{itemize}
\end{frame}

%-------------------------------------------------------------------------------

\subsubsection{Windows}
\begin{frame}
\frametitle{Introduction}
\framesubtitle{Free Editors -- Windows}
\begin{itemize}
\item TeXnicCenter\footnote{http://www.texniccenter.org/}
  %\item \ldots
\end{itemize}
\end{frame}

%-------------------------------------------------------------------------------

\subsubsection{Cross-Platform}
\begin{frame}
\frametitle{Introduction}
\framesubtitle{Free Editors -- Cross-Platform}
\begin{itemize}
  \item TeXMaker\footnote{http://www.xm1math.net/texmaker}\\
   Used in this turorial.
  \item TeXstudio\footnote{http://sourceforge.net/projects/texstudio/?source=dlp}\\
  Like TeXMaker, but more powerful.
  \item TeXlipse\footnote{http://texlipse.sourceforge.net/}\\ For advanced users, plugin for
  Eclipse. Good IDE support, code-completion, autobuilds, version control etc.
\end{itemize}
\end{frame}

%-------------------------------------------------------------------------------

\begin{frame}
\frametitle{TeXmaker}
\framesubtitle{Overview}
\image{\textwidth}{image/texmaker_overview.png}{Standard window of the Texmaker}{img:texmaker1}

\end{frame}

%-------------------------------------------------------------------------------

\begin{frame}
\frametitle{TeXmaker}
\framesubtitle{Synctex}
\image{\textwidth}{image/synctex.png}{Synctex}{img:synctex}

\end{frame}

\begin{frame}
\frametitle{My very first \LaTeX -document}
\begin{block}{New commands:}
\begin{itemize}
\item \begin{ttfamily}\color{nounibaredII}\textbackslash documentclass\color{nounibagreenI}\color{black}\{article\}\end{ttfamily}
\item \begin{ttfamily}\color{unibablueI}\textbackslash begin\color{black}\{document\}\end{ttfamily}
\item \begin{ttfamily}\color{unibablueI}\textbackslash end\color{black}\{document\}\end{ttfamily}
\end{itemize}
\end{block}
This is all you need for a \LaTeX -document. Let's try!

\end{frame}
