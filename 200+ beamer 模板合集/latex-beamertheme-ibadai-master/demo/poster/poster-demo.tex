%%%%%% \documentclass[%10pt,compress,
%%%%%% %              draft,
%%%%%%                xcolor={dvipsnames,table},
%%%%%%                hyperref={breaklinks}
%%%%%%               ]{beamer}

\documentclass[final]{beamer}
%%%%%
% Fonts, input, language and other nice things
% http://tex.stackexchange.com/q/44694/33413
% http://tex.stackexchange.com/q/664/33413
% 
\usepackage[T1]{fontenc} 
\usepackage[utf8]{inputenc}
% Use a nicer font
\usepackage{lmodern}
% Wrong scaling from lmodern
% http://www.tex.ac.uk/FAQ-exscale.html
\usepackage{exscale}
% Hyphenation et.al.
\usepackage[english]{babel}
% Non-italised greek letters
\usepackage[euler]{textgreek}
% Roman numerals
\usepackage{romanbar}                           %Römische Zahlen mit Balken.

%%%%%
% Science related packages
%
%%%
% Mathematics
%
\usepackage{amsmath}
\usepackage{amssymb}
%%%
% Units
%
% Easy typesetting of SI conform units
\usepackage{siunitx}
\DeclareSIUnit  % Use non-SI units
  \calorie{cal} % calorie, i.e. kcal/mol
\DeclareSIUnit  
  \electron{e}  % electron, i.e. eV
%%%
% Chemistry
%
% Easy typesetting of formulae
\usepackage[version=4,arrows=pgf-filled]{mhchem}
% Enumerating compounds
\usepackage{chemnum}
% Support for creating and using schemes
% http://tex.stackexchange.com/q/6478/33413
\usepackage{newfloat}
\DeclareFloatingEnvironment[
  fileext=los,
  listname=List of Schemes,
  name=Scheme,
  placement=tbhp,
%  within=section,
]{scheme}
%%%
% Misc.
%
% Manual set-up for captions
% this is predominantly handy if used for posters
\usepackage[%font+=footnotesize, %small,
            %labelfont+={footnotesize,bf},
            format=hang,
            singlelinecheck=no, %for poster
            justification=justified %for poster
           ]{caption}
% provides captions comand without using environments
\usepackage{capt-of}
% Nice tables
\usepackage{booktabs}

%%%%%
% Graphic settings
% http://tex.stackexchange.com/q/23075/33413
% http://tex.stackexchange.com/q/139401/33413
% 
\usepackage{graphicx}
\graphicspath{{./graphics/}}
% http://ctan.org/pkg/adjustbox
% Using the export key with adjustbox, this will load the graphicx package, 
% and allow you to use its keys as part of \includegraphics.
\usepackage[export]{adjustbox}                        
% Use animations within presentation
\usepackage{animate}
                                                      
%%%%%
% Beamer specific commands
% possible options are alttitle, light, dark (default), poster
% specifying the latter uses its own templates,
% so the former options have no effect
%
\usetheme[poster]{Mito}
%\usefonttheme[stillsansseriflarge]{serif}
%\usefonttheme[onlymath]{serif}

% prevent counting the appendix as slides
% \usepackage{appendixnumberbeamer} %only beamer mode
% Setting title, etc.
\title{Short and Catchy Title}
\subtitle{Long and boring subtitle with unnecessary explanations.}
\author[F. Bar]{Foo Bar}
\institute[Baz Inst.]{Baz Insititute}
%\titlegraphic{\includegraphics[scale=2]{example-image-1x1}}
\titlegraphic{\includegraphics[scale=2]{example-image}}
% example-image from https://www.ctan.org/pkg/mwe
% need to be scaled up, because they appear too small
\date{the Internet, \today}
% The following command can be used to create a custom footline for 
% a poster of the alternative titlepage.
% This is optional, but it will be set with default content.
\posterfootline{\insertdate\hfill Very Important Symposium}
% The following command can be used to give extra content to the alt. titlepage.
% This is optional and won't be set if empty.
% It has no use for a poster. 
% \titlepageextra{\includegraphics[width=0.5\linewidth]{example-image}}
%
% Make a poster, e.g. for DIN-A0 poster
\usepackage[orientation=portrait,size=a0,scale=1.4,debug]{beamerposter}

% Use multiple columns, the beamer built-in is only okay-ish
\usepackage{multicol}

%%%%%
% Custom commands and hacks
%
% Only for this demonstration we like
\usepackage{blindtext}
%
% custom newcommands can go here, for example
\newcommand{\diff}{\mathrm{d}} % upright differential operator
%
%%%%% End preamble.

\begin{document}

% All content has to fit into one frame
\begin{frame}[t]
\setlength{\columnsep}{0.02\textwidth}
\begin{multicols}{2}
  \begin{block}{A normal block with text and lists}
    Each block contains a minipage environment so it can be set as a normal page.
    This includes footnotes\footnote{See here} at the bottom of the block.

    Citations are supported in the freeform \texttt{thebibliography} environment
    and can be inserted via \texttt{\textbackslash{}cite\{<key>\}}. 
    This example uses the \texttt{[text]} template, see \cite{beamerhandbook} 
    for more information.

    It is possible to use more advanced citation handling like \texttt{biblatex},\cite{tex.sx}
    but this template is not yet finished, so it'll have to wait.

    \blindtext
    \blindlistlist[3]{itemize}[3]

    \blindlistlist[3]{enumerate}[2]
  \end{block}
  
  \begin{block}{Many figures}
    \blindtext

    \begin{center}
      \includegraphics[width=0.4\linewidth]{example-image-a}\hspace{0.1\linewidth}
      \includegraphics[width=0.4\linewidth]{example-image-b}
      \captionof{figure}{This is an example image from the mwe package.}
      \includegraphics[width=0.4\linewidth]{example-image-c}
      \captionof{figure}{This is an example image from the mwe package.}
    \end{center}
  \end{block}

  \begin{block}{And a table of course}
    \blindtext
%     Hello, here is some text without a meaning. This text should show what a
%     printed text will look like at this place. If you read this text, you will get no in-
%     formation. Really? Is there no information? Is there a difference between this
%     text and some nonsens?

    \begin{center}
      \captionof{table}{This is an example table.}
      \begin{tabular}{lrrrr}
        \toprule
               & Column A & Column B & Column C & Column D  \\
        \midrule                                          
        Row 1  & A1       & B1       & C1       & D1        \\
        Row 2  & A2       & B2       & C2       & D2        \\
        Row 3  & A3       & B3       & C3       & D3        \\
        Row 4  & A4       & B4       & C4       & D4        \\
%        Row 5  & A5       & B5       & C5       & D5        \\
        \bottomrule
      \end{tabular}
    \end{center}
  \end{block}
  
  \begin{block}{Mathematics}
    Mathematics can be included in-line or as a display. 
    For the following examples we can substitute \(u=e^x\) and \(v=e^{-x}\) 
    to save us some writing pain.
    
    The fundamental tricks of algebra are adding zero \eqref{eq:addingzero} %
    or multiplying by one \eqref{eq:multiplyingone}.
    \begin{subequations}
      \begin{align}\label{eq:addingzero}
        \int \frac{\diff x}{1 + e^x} &= \int \frac{\diff u}{u (u + 1)}\\ 
          &= \int \diff u \frac{1 + u - u}{u (1 + u)} \\
          &= \int \frac{\diff u}{u} - \int \frac{\diff u}{1 + u}
          &&= \ldots
      \end{align}
    \end{subequations}
    \begin{subequations}
      \begin{align}\label{eq:multiplyingone}
        \int \frac{\diff x}{1 + e^x} &= \int \diff x \frac{e^{-x}}{1 + e^{-x}} \\
          &= -\int \frac{\diff v}{1 + v}
          &&= \ldots
      \end{align}
    \end{subequations}
  \end{block}

  \begin{block}{Chemistry}
    Writing about chemistry can be a tricky, but additional packages can make it easier.
    It is similar to mathematics and can be quite easily incorporated into it.
  
    For example take the combustion of dihydrogen (\ce{H2}, \cmpd{dihydrogen})
    with dioxygen (\ce{O2}, \cmpd{dioxygen}) to form water (\ce{H2O}, \cmpd{water}).\cite{chem.se}
    The reaction seems quite simple: \[\ce{H2 + O2 -> 2 H2O}.\]
  
    The mechanism is much more complex and even scheme~\ref{sch:mechanism} is incomplete.
  
    \begin{align*}
      \ce{H2 &<=>[\Delta T] 2 H.}\\
      \ce{H. + O2 &-> HO. + O.}\\
      \ce{O. + H2 &-> HO. + H.}\\
      \ce{HO. + H. &-> H2O}
    \end{align*}
    \captionof{scheme}{Incomplete reaction mechanism of the combustion of \cmpd{dihydrogen} and \cmpd{dioxygen}.}
    \label{sch:mechanism}
  \end{block}

  \begin{block}{Bibliography (needs work)}
    \begin{thebibliography}{99}
      \bibitem{example} A~Author~et.al., *Journal* **year,** *vol.*,  pp.
      \bibitem{beamerhandbook} Beamer User Guide, Cahpter 10.6
      \bibitem{tex.sx} http://tex.stackexchange.com/q/69133/
      \bibitem{chem.se} http://chemistry.stackexchange.com/q/14704/
    \end{thebibliography}
    %\blindtext
  \end{block}
\end{multicols}

Rather unimportant information can be gathered below the columns,
just before the footline. \dotfill \LaTeX
\end{frame}

\end{document}

