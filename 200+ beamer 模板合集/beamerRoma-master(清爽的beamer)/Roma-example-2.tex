\documentclass{beamer}
\usepackage[italian]{babel}

% To typeset with pdflatex decomment the following 2 lines
%\usepackage[utf8]{inputenc}
%\usepackage[T1]{fontenc}
% To typeset with lualatex decomment the following line
\usepackage{fontspec}

\usepackage{appendixnumberbeamer}

\hypersetup{pdfpagemode=FullScreen}

\title[Tesi complicata]{Tesi complicata sui Particelloni di Rubinenstein}
\subtitle[Tesi Magistrale]{Tesi Magistrale in Fisica - Fisica delle Merendine}
\author[Mario]{Mario Rossi - \texttt{mariorossi@roma.edu} \\
relatore: Prof. Dario Lampa}
\date{01 ottobre 2012}
%\institute[Roma]{Università di Roma}
\institute[Sapienza Università di Roma]{
  Dipartimento di Fisica\\
  Sapienza Università di Roma}

% PDF information catalog
\subject{Tesi Magistrale in Fisica - Fisica delle Merendine}

\usetheme[pageofpages=of, % default - any character is good: di, /
          titleline=true, % default - other choice: false
          ]{Roma}

\theoremstyle{definition}
\newtheorem{definizione}{Definizione}

\theoremstyle{plain}
\newtheorem{teorema}{Teorema}

\begin{document}
\titlepageframe % use this command to create the titlepage

\section*{Indice}
\begin{frame}
  \frametitle{Indice}
  \tableofcontents
  % You might wish to add the option [pausesections]
\end{frame}

\section[Numeri casuali]{Generazione di Numeri casuali}
\begin{frame}
  \frametitle{Sequenze aleatorie}
  \begin{block}{Che cos’è un numero casuale?}
    Il problema è mal posto. Esistono le \alert{sequenze} di numeri aleatori!
  \end{block}
  \begin{block}{Sequenze di numeri casuali}
    \begin{itemize}
	  \item veramente casuali
      \item pseudo-casuali
      \item quasi-casuali
    \end{itemize}
  \end{block}
\end{frame}

\section[Test]{Test statistici}
\begin{frame}
  \frametitle{Perché sono stati sviluppati?}
  \begin{block}{Una sequenza pseudo-casuale}
    \begin{itemize}
      \item deve seguire la distribuzione desiderata
      \item deve essere non correlata
    \end{itemize}
  \end{block}
  \begin{block}{Problemi}
    \begin{itemize}
      \item distribuzione non conforme a quella attesa
      \item sequenza fortemente correlata
      \item clustering dimensionale
    \end{itemize}  
  \end{block}
\end{frame}

\section*{Conclusioni}
\begin{frame}
  \frametitle{Conclusioni}
  \begin{block}{Questo tema è}
    \begin{itemize}
      \item bello
      \item semplice
      \item leggibile
      \item da finire
      \item per la Sapienza
    \end{itemize}
  \end{block}
\end{frame}

\appendix
\section{\appendixname}
\frame{\tableofcontents}

\subsection{Additional material}
\frame{Details}
\frame{Text omitted in main talk.}

\subsection{Even more additional material}
\frame{More details}

\end{document}
