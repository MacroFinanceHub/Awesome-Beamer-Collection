\documentclass{beamer}
\usepackage[italian]{babel}

% To typeset with pdflatex decomment the following 2 lines
\usepackage[utf8]{inputenc}
\usepackage[T1]{fontenc}
% To typeset with lualatex decomment the following line
%\usepackage{fontspec}

\title{I numeri primi sono infiniti}
\author[Euclide]{Euclide di Roma \\
\texttt{euclide@roma.edu}}
\date[VII SINP]{VII Simposio Internazionale sui Numeri Primi}
\institute[Roma]{Università di Roma}

\usetheme[pageofpages=of, % default - any character is good: di, /
          titleline=true, % default - other choice: false
          ]{Roma}

\theoremstyle{definition}
\newtheorem{definizione}{Definizione}

\theoremstyle{plain}
\newtheorem{teorema}{Teorema}

\begin{document}
\titlepageframe % use this command to create the titlepage


\begin{frame}
  \frametitle{Piano della presentazione}
  \tableofcontents
\end{frame}

\AtBeginSection[]
{
\begin{frame}<beamer>
\frametitle{Outline}
\tableofcontents[currentsection]
\end{frame}
}

\section{Introduzione}
\begin{frame}
\frametitle{Che cosa sono i numeri primi?}
\begin{definizione}
Un \alert{numero primo} è un intero $>1$ che ha esattamente
due divisori positivi.
\end{definizione}
\end{frame}

\section{L'infinità dei primi}
\subsection{La dimostrazione}
\begin{frame}
\frametitle{I numeri primi sono infiniti}
\framesubtitle{Ne diamo una dimostrazione diretta}
\begin{teorema}
Non esiste un primo maggiore di tutti gli altri.
\end{teorema}
\pause
\begin{proof}
\begin{enumerate}[<+->]
\item Sia dato un elenco di primi.
\item Sia $q$ il loro prodotto.
\item Allora $q+1$ è divisibile per un primo $p$
che non compare nell’elenco. \qedhere
\end{enumerate}
\end{proof}
\end{frame}
\section{Problemi aperti}
\begin{frame}
\frametitle{Che cosa c’è ancora da fare?}
\begin{block}{Problemi risolti}
Quanti sono i numeri primi?
\end{block}
\begin{alertblock}{Problemi aperti}
Un \alert{numero} pari $>2$ è sempre la somma di due primi?
\end{alertblock}
\end{frame}


\end{document}
