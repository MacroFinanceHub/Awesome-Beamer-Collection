\lecture{使用说明}{lec:introduction}
\section{简介}
% 这一主题的动机
\begin{frame}{简介}{内容简介}
  \begin{itemize}
  \item 西北农林科技大学 {\LaTeX} \alert{Beamer 主题}
    \begin{itemize}
    \item 简单易用
    \item 标准、规范
      \begin{itemize}
      \item 高质量
      \item 高效率
      \end{itemize}
    \item \alert{所想即所得}
    \end{itemize}
  \end{itemize}
\end{frame}
%%%%%%%%%%%%%%%%

\subsection{协议}
% 协议
\begin{frame}{简介}{协议}
  \begin{itemize}
  \item 所有logo的版权属于西北农林科技大
    学\href{http://www.nwafu.edu.cn}{http://www.nwafu.edu.cn}
  \item 若署名为西北农林科技大学,则可以使用这些logo
  \item 请遵守 GNU 通用公共协议 v.3 (GPLv3)
    \begin{itemize}
    \item 详
      见:
      \href{http://www.gnu.org/licenses/}{http://www.gnu.org/licenses/}
    \item 可以发布和修改本主题中的任何内容
    \end{itemize}
  \end{itemize}
\end{frame}
%%%%%%%%%%%%%%%%

\section{安装}
% 通用安装
\begin{frame}{安装}{概述}
  \begin{itemize}
  \item 文件构成
    \begin{enumerate}
    \item {\tt beamerthemenwafusidebar.sty}
    \item {\tt beamerinnerthemenwafusidebar.sty}
    \item {\tt beamerouterthemenwafusidebar.sty}
    \item {\tt beamercolorthemenwafusidebar.sty}
    \item {\tt beamerfontthemenwafusidebar.sty}
    \end{enumerate}
  \item 全局安装
    \begin{itemize}
    \item 拷贝到本地\LaTeX 目录树中
    \end{itemize}
  \item 本地安装
    \begin{itemize}
    \item 将5个主题文件拷贝到当前工作文件夹
    \end{itemize}    
  \end{itemize}
\end{frame}

\subsection{GNU/Linux}
% GNU/Linux中的安装
\begin{frame}{安装}{GNU/Linux}
  \begin{block}{Ubuntu中的TeX Live}
    \begin{enumerate}
    \item 将 {\tt <dirstruct>} 拷贝到本地{\LaTeX}目录树的根目录. 默认是\\
      {\tt \textasciitilde /texmf}\\
      如果根目录不存在,则创建该目录。 符号 {\tt \textasciitilde} 表示
      家目录, 例如:{\tt /home/<username>}
    \item 在终端中运行如下命令\\
      {\tt \$ texhash \textasciitilde /texmf}
    \end{enumerate}
  \end{block}
\end{frame}
%%%%%%%%%%%%%%%%

% Microsoft Windows
\begin{frame}{安装}{Microsoft Windows}
  \begin{block}{Windows中的TeX Live}
    假设使用默认目录(在高级 TeX Live 安装中,可以更改latex目录树的根目
    录)。
    \begin{enumerate}
    \item 将 {\tt <dirstruct>} 拷贝到本地{\LaTeX}目录树的根目录\\
      {\tt \%USERPROFILE\%\textbackslash texmf}\\
      如果不存在,则创建. XP的默认目录是 {\tt \%USERPROFILE\%} 是\\
      {\tt c:\textbackslash Document and
        Settings\textbackslash<username>},\\
      Vista及更高版本是\\
      {\tt c:\textbackslash Users\textbackslash<username>}
    \item 打开 TeX Live 管理器对话框选择 'Actions'中的'Update filename
      database',并执行.
    \end{enumerate}
  \end{block}
\end{frame}
%%%%%%%%%%%%%%%%

\subsection{Mac OS X}
% Mac OS X的安装
\begin{frame}{安装}{Mac OS X}
  \begin{block}{Mac OS X中的 MacTeX}
    将 {\tt <dirstruct>} 拷贝到本地latex目录树的根目录. 默认是\\
    {\tt \textasciitilde /Library/texmf}\\
    如果不存在,则创建. 符号 {\tt \textasciitilde} 表示家目录, 例
    如:{\tt /home/<username>}
  \end{block}
\end{frame}
%%%%%%%%%%%%%%%%

\subsection{宏包依赖}
% 宏包依赖
\begin{frame}{安装}{宏包依赖}
  除需要 Beamer 类外,本主题需要调用两个宏包
  \begin{itemize}
  \item TikZ\footnote{TikZ 是一个绘制图形的杰出宏包.请参
      考\href{http://www.texample.net/tikz/examples/}{在线示
        例} 或
      \href{http://tug.ctan.org/tex-archive/graphics/pgf/base/doc/generic/pgf/pgfmanual.pdf}{pgf
        用户手册}. }
  \item calc
  \end{itemize}
  这些宏包是{\LaTeX}的通用宏包。
\end{frame}
%%%%%%%%%%%%%%%%

\section{用户接口}
\subsection{主题及选项}
% 主题和选项列表
\begin{frame}{用户接口}{加载主题和主题选项}
  \begin{block}{演示文稿主题}
    加载主题只需要输入\\
    {\tt \textbackslash usetheme[<选项>]\{nwafusidebar\}}\\
    与加载其它主题方法一致,本主题会加载内部、外部和颜色主题并且可以传
    递 {\tt <选项>} 参数.
  \end{block}
  \begin{block}{内部主题}
    使用如下命令加载内部主题\\
    {\tt \textbackslash useinnertheme\{nwafusidebar\}}\\
    内部主题无参数.
  \end{block}
\end{frame}
%%%%%%%%%%%%%%%%

\begin{frame}{用户接口}{加载主题和主题选项}
  \begin{block}{外部主题}
    使用如下命令加载外部主题\\
    {\tt \textbackslash useoutertheme[<选项>]\{nwafusidebar\}}\\
    目前,外部主题的参数有:
    \begin{itemize}
      \scriptsize
    \item {\tt hidetitle}: 隐藏边栏中的短标题
    \item {\tt hideauthor}: 隐藏边栏中的作者缩写
    \item {\tt hideinstitute}: 隐藏边栏底部的单位缩写
    \item {\tt shownavsym}: 显示导航符号
    \item {\tt left} or {\tt right}: 边栏位置 (默认在右边)
    \item {\tt width=<length>}: 边栏宽度 (默认是 2 cm)
      % 宽度指从垂直分割条的右边到slide的右边
    \item {\tt hideothersubsections}: 除了当前section的subsection隐藏其
      它所有 subsections
    \item {\tt hideallsubsections}: 隐藏所有 subsections
    \end{itemize}
    最后4个选项继承于外部sidebar主题.
  \end{block}
\end{frame}
%%%%%%%%%%%%%%%%
\subsection{编译}
% 编译
\begin{frame}{用户界面}{编译}
  \begin{block}{演示文稿的编译}
    本主题需要至少编译 \alert{3} 次,以保证正确处理页码计数器的数字。
  \end{block}
\end{frame}
%%%%%%%%%%%%%%%%

\subsection{主题修改}
% 如何修改主题
{\setbeamercolor{nwafusidebar}{fg=gray!50,bg=gray}
  \setbeamercolor{sidebar}{bg=red!20}
  \setbeamercolor{structure}{fg=red}
  \setbeamercolor{frametitle}{use=structure,fg=structure.fg,bg=red!5}
  \setbeamercolor{normal text}{bg=gray!20}
  \begin{frame}{用户界面}{主题修改}
    \begin{itemize}
    \item 主题设置了默认的字体、颜色和布局。
    \item 请参考beamer用户手册。
    \item 例如,在这一页中,使用如下方式修改了主题元素
      \begin{itemize}
      \item 修改边栏颜色:\\
        {\tt \textbackslash
          setbeamercolor\{nwafusidebar\}\{fg=gray!50,bg=gray\}} {\tt
          \textbackslash setbeamercolor\{sidebar\}\{bg=red!20\}}
      \item 修改结构元素颜色:\\
        {\tt \textbackslash setbeamercolor\{structure\}\{fg=red\}}\\
      \item 修改帧标题文本颜色和背景颜色: {\tt \textbackslash
          setbeamercolor\{frametitle\}\{use=structure,
          fg=structure.fg,bg=red!5\}}
      \item 修改文本背景颜色{\tt \textbackslash setbeamercolor\{normal
          text\}\{bg=gray!20\}}
      \end{itemize}
    \end{itemize}
  \end{frame}}
%%%%%%%%%%%%%%%%

\subsection{波浪背景}
% nwafu波浪背景图案
\begin{frame}{用户界面}{波浪背景图案}
  \begin{block}{波浪背景图案}
    \begin{itemize}
    \item 可以在任意一个单独帧中用下述方法添加背景图案\\
      {\tt \{\textbackslash nwafuwavesbg\\
        \textbackslash begin\{frame\}[<选项>]\{帧标题\}\{帧子标题\}\\
        \ldots\\
        \textbackslash end\{frame\}\}}
    \end{itemize}
  \end{block}
\end{frame}
%%%%%%%%%%%%%%%%

\subsection{宽屏支持}
% 宽屏支持
\begin{frame}{用户界面}{宽屏支持}
  \begin{block}{宽屏支持}
    新式投影仪,甚至是现代的电视机已支持如 16:10 或 16:9 模式的宽屏模式.
    Beamer (>= v. 3.10) 支持多种演示文稿的缩放比例。根据Beamer 用户手册
    (v. 3.10)的77页的8.3节的说明,可以使用如下选项设置显示比例\\
    {\tt\textbackslash documentclass[aspectratio=1610]\{beamer\}}\\
    这一命令设置为 16:10. 也可以是 169, 149, 54, 43 (默认).
  \end{block}
\end{frame}


%%%%%%%%%%%%%%%%

\section{主题应用}
{\setbeamercolor{nwafusidebar}{fg=gray!50,bg=gray}
 \setbeamercolor{sidebar}{bg=red!20}
 \setbeamercolor{structure}{fg=red}
 \setbeamercolor{frametitle}{use=structure,fg=structure.fg,bg=red!5}
 \setbeamercolor{normal text}{bg=gray!20}
 \subsection{文件夹结构}
\begin{frame}{主题应用}{文件夹结构}
  \begin{itemize}
  \item 本地安装时的工作文件构成
  \end{itemize}
  \begin{center}
    \scalebox{0.58}{
      \begin{forest}
        pic dir tree,
        pic root,
        for tree={% folder icons by default; override using file for file icons
          directory,
        },
        [jobname【工作根目录】%,           
          [contents【分章节内容\LaTeX 源文件,可根据需要增减】        
            [ch01.tex【第1章】, file
            ]
            [$\vdots$, file
            ]
          ]
          [nwafulogo【学校校徽图标,\alert{必须存在},且置于根目录】
            [circle.pdf【学校圆形透明Logo】, file
            ]
            [h\_bar.pdf【学校横条形中英文透明Logo】, file
            ]
            [nwafu\_waves.pdf【波纹背景透明图】, file
            ]
            [nwafu\_logo\_cie.png【信息工程学院透明Logo】, file
            ]
          ]
          [settings【自定义命令、环境、引入宏包等\LaTeX 源文件,可根据需要调整】        
            [format.tex【自定义命令、环境、参数设置等】, file
            ]
            [packages.tex【引入宏包】, file
            ]
          ]
          [beamerthemenwafusidebar.sty【主题主文件,\alert{必须存在},且置于根目录】, file
          ]
          [beamerinnerthemenwafusidebar.sty【内部主题文件,\alert{必须存在},且置于根目录】, file
          ]
          [beamerouterthemenwafusidebar.sty【外部主题文件,\alert{必须存在},且置于根目录】, file
          ]
          [beamercolorthemenwafusidebar.sty【颜色主题文件,\alert{必须存在},且置于根目录】, file
          ]
          [beamerfontthemenwafusidebar.sty【字体主题文件,\alert{必须存在},且置于根目录】, file
          ]
          [main.tex【主控文件,\emph{必须存在},且置于根目录】, file
          ]
          [Makefile【make命令需要的脚本文件,若不执行make命令,可以不需要】, file
          ]
          [.latexmkrc【latexmk命令需要的脚本文件,若不执行latexmk命令,可以不需要】, file
          ]
        ]
      \end{forest}
    }
  \end{center}
\end{frame}
}
%%%%%%%%%%%%%%%%

\section{问题反馈}
\subsection{错误, 意见和建议}
% 问题、意见和建议
\begin{frame}{问题反馈}{错误、意见和建议}
  \begin{itemize}
  \item 本主题中会存在错误,如果发现了错误,请联系我进行改进
    \begin{itemize}
    \item \alert{再小的问题也是问题}
    \end{itemize}
  \item 如果你有好建议和使用体验改善意见,请联系我进行改进
  \end{itemize}
\end{frame}
%%%%%%%%%%%%%%%%

% \tiny
% \scriptsize
% \footnotesize
% \small
% \normalsize
% \large
% \Large
% \LARGE
% \huge
% \Huge

%%% Local Variables: 
%%% mode: latex
%%% TeX-master: "../main.tex"
%%% End: 
