
\documentclass{beamer}


%
%  Options list:
%
%   strictfont - use Sandia Corporate-specified font (Calibri).  Omit if you don't
%      care about using Calibri or don't have it installed.
%   theme - select theme.  Choose from blue,white,sand,left,right.  Try it!
%   slidenumber - include slide numbers on body slides
%   slidedate - include date on body slides
%   titledate - include date on title slide
%


\mode<presentation>
{
  \usetheme[strictfont,theme=blue,slidenumber,slidedate]{Sandia}
}


\title{A Wonderful Sample Slidedeck}
\subtitle{This is the subtitle}
\author{Patrick M. Widener}

% 
%  Use this macro to turn on OUO markings and provide the appropriate OUO information
%  Options include:
%    exemption - numeric (2-9) indicator of which FOIA exemption this material falls under
%       The template will supply the text.
%    nameorg - OUO name and organization, as a string
%    date - OUO date, defaults to \insertdate
%    guidance - Any OUO guidance to include on the marking
%
%\sandiabeamerMarkOUO{exemption=3,nameorg=Patrick M. Widener (1423)}

%
% Use this macro to specify the SAND number of your presentation
%
\sandiabeamerSANDInfo{number=2014-00000}

\begin{document}

%
%  Trivia: the content in this sample is from an intro-to-HPC class I taught back in 2011, 
%  so some of it is dated by this point.  It serves the purpose, though, of filling out 
%  this sample slidedeck.
%

\begin{frame}
\titlepage
\end{frame}

\section{Performance principles}

\begin{frame}
\frametitle{Performance levels}

\begin{itemize}
\item Peak performance
\begin{itemize}
\item Summ of all speeds of all FP units in system
\item Theoretical upper bound on performance
\end{itemize}

\item LINPACK
\begin{itemize}
\item The ``Hello World'' of parallel performance codes
\item Solve Ax=B using Gaussian elimination, highly tuned
\end{itemize}

\item Gordon Bell Prize-winning application performance
\begin{itemize}
\item Right application + right algorithm + right platform + years of
  effort
\end{itemize}

\item Average sustained application performance
\begin{itemize}
\item What one can reasonably expect for standard applications
\end{itemize}

\end{itemize}

\begin{beamerboxesrounded}{When reporting performance results, these levels are often confused,
even in reviewed publications}
\end{beamerboxesrounded}
\end{frame}

\begin{frame}
\frametitle{Performance levels (NERSC-5)}

\begin{itemize}
\item Peak advertised performance: {\color{red} 100 Tflop/s}
\item LINPACK (TPP): {\color{red} 84 Tflop/s}
\item Best climate application: {\color{red} 14 Tflop/s}\\
WRF code benchmarked in December 2007
\item Average sustained application performance: {\color{red} ??}\\
Probably less than 10\% peak
\end{itemize}
\end{frame}

\begin{frame}
\frametitle{Logistics}

\begin{itemize}

\item Web page and/or Google group and/or...
\begin{itemize}
\item Depends on class composition and preference
\item Reading list posted this weekend
\item Round-robin student presentations and group discussion
\end{itemize}


\item Some lecture on certain topics

\item Guest lectures

\item Hands-on
\begin{itemize}
\item 3 or 4 programming assignments
\item First assignment Monday:  warmup / refresh on concurrency /
  multithreading
\item CCI cluster hopefully available, CS cluster
\end{itemize}

\item Term project:  larger scale
\begin{itemize}
\item Propose something, or I have a few candidates
\item Make this effort do double duty if possible
\item Implementation and performance evaluation
\item Teams OK if we have enough people and scope is right
\end{itemize}

\end{itemize}
\end{frame}

\begin{frame}
\frametitle{List of topics}

Subject to change as we go along:
\begin{itemize}
\item Architectures: performance characteristics of parallel machines,
  shared-memory (NUMA, multicore) and distributed memory
\item Applications: scientific and engineering applications
\item Decomposition methods - domain, functional, pipelining,
  divide/conquer
\item Programming and OS constructs and models (emphasis)
\item Parallel I/O and storage systems
\item Interconnection networking
\item Lightweight Kernels
\item Virtualization
\end{itemize}

\end{frame}


\end{document}

%%% Local Variables: 
%%% mode: latex
%%% TeX-master: t
%%% End: 
