\documentclass[english]{beamer}
\usepackage{babel}

% Load system fonts Open Sans and Courier (via fontspec)
% \usepackage{fontspec}
% \setmainfont{Open Sans}
% \setsansfont{Open Sans}
% \setmonofont{Courier}

% Load LaTeX font package 'Helvetica'
\usepackage[utf8]{inputenc}
\usepackage{helvet}

\usepackage{listings}

\usetheme{upb}

\title{Beamer Theme UPB}
\subtitle{Unofficial Theme of the University of Paderborn}
\author{Michael Knopf}


\lstset{
	basicstyle=\ttfamily\footnotesize,
	identifierstyle=\color{upbblue},
	commentstyle=\itshape\color{upbgreen},
	frame=leftline,
	framerule=3pt,
	rulecolor=\color{upbblue!30},
	backgroundcolor=\color{upbblue!10},
	numbers=left,
	numberstyle=\tiny\bfseries\itshape\color{upbblue},
	tabsize=2
}

\begin{document}
	
\frame{
	\maketitle
}

\section{Introduction}


\frame{
	\frametitle{An Unofficial Theme}
	
	This is an unofficial theme of the \emph{University of Paderborn} for the {\LaTeX} Beamer class.
	
	\vfill	
	
	\begin{block}{Features and Design Goals}
		\begin{itemize}
			\item Unobtrusive but modern design
			\item Compliance with corporate design guidelines
			\item Good printability
		\end{itemize}
	\end{block}
	
	\vfill
	
	The latest version of the theme is available at:\
	\url{https://bitbucket.org/jimknopf/beamerthemeupb}
}



\section{Usage \& Features}
\frame[containsverbatim]{
	\frametitle{Suitable Font Families}

	Examples of suitable {\LaTeX} packages are \texttt{\color{upbgreen}helvet} (\emph{Helvetica}) and \texttt{\color{upbgreen}droid} (available in most modern distributions).
	
	\begin{lstlisting}[language=tex]
% UTF-8 encoding and 'droid' font family
\usepackage[utf8x]{inputenc}
\usepackage{droid}
	\end{lstlisting}

	\vfill

	The \texttt{\color{upbgreen}fontspec} package is a modern alternative that allows the inclusion of system fonts.
	
	\begin{lstlisting}[language=tex]
% Use 'fontspec' to load system fonts
\usepackage{fontspec}
\setsansfont{Open Sans}
\setmonofont{Courier}
	\end{lstlisting}


\vfill

This presentation uses \emph{Open Sans} and \emph{Courier}!
}

\frame{
	\frametitle{Additional Color Definitions}
	Besides the \emph{official} blue of the university there are definitions for matching red and green tones.
	
	\begin{center}
	\tikzstyle{every node} = [circle, minimum size = 2.5cm]
	\begin{tikzpicture}
		\draw[use as bounding box, draw=none] (-1.25,1.25) rectangle (7.25,-3.25);
			\foreach \x/\c in {0/upbred, 3/upbgreen, 6/upbblue} {
				\foreach \y/\p in {2/20, 1.5/40, 1/60, 0.5/80} {
					\draw (\x, -\y) node[fill=\c!\p] {};
				};
			};
		
		\draw (0,0) node[text=white,fill=upbred] {\emph{upbred}};
		\draw (3,0) node[text=white,fill=upbgreen] {\emph{upbgreen}};
		\draw (6,0) node[text=white,fill=upbblue] {\emph{upbblue}};
		
		\draw (7.15,-4.375) node[rotate=-6,text=upbblue,fill=upbblue!30,minimum size = 1.5cm] {\tiny \textbf{\emph{upbblue!30}}};
		
	\end{tikzpicture}
	
\end{center}
}

\logo{\pgfimage[width=3.25cm]{upb_logo}}
\frame[containsverbatim]{
	\frametitle{Working Group Logos}
	
	Inclusion of a second logo:
	\begin{itemize}
		\item Support of the \lstinline[language=tex]$\logo{...}$ command
		\item Placement in the upper right corner
	\end{itemize}
	
	\vfill
	
	Example (used on this slide):
	\begin{lstlisting}[language=tex]
% Include external figure as logo
% (this example includes the university logo twice)
\logo{\pgfimage[width=3.25cm]{upb_logo.pdf}}
	\end{lstlisting}
	
	\vfill
	
	It is not recommended to replace the logo in the footer!
}

\logo{}
\frame[containsverbatim]{
	\frametitle{Sections}

	Sections can be used to indicate the progress of a talk:
	\begin{itemize}
		\item The \emph{current} section is highlighted
		\item Anonymous sections ( \lstinline[language=tex]$\section*{...}$)  and subsections are ignored
	\end{itemize}
	
	\vfill
	
	\begin{exampleblock}{Tip}
		Do not use the  \lstinline[language=tex]$\section{...}$ command before the title page: this way no section will be highlighted at the beginning of your talk!
	\end{exampleblock}		

}

\section{Template}

\frame[containsverbatim]{
	\frametitle{Template (I/II)}
	A simple template using a suitable font family and providing a basic structure (note the use of sections) \dots 
	
	\vfill
	
	\begin{lstlisting}[language=tex]
% Beamer presentation (english)
\documentclass{beamer}
\usepackage[english]{babel}
% UTF-8 encoding and LaTeX font package 'helvet'
\usepackage[utf8x]{inputenc}
\usepackage{helvet}
% Use the UPB theme
\usetheme{upb}
% If provided the short title is shown in the footer,
% otherwise the long title is included
\title[Short Title (optional)]{Long Title}
\subtitle[Subtitle (optional)]
\author{My Name}
\date{\today}
	\end{lstlisting}
}

\frame[containsverbatim]{
	\frametitle{Template (II/II)}
	\begin{lstlisting}[language=tex, firstnumber=15]
\begin{document}
	% Introductionary slide (not assigned to any section)
	\frame{\maketitle}
	\frame{
		\frametitle{Introduction}
	}
	% First section with content
	\section{First Section}
	\frame{...}
	\frame{...}
	% ...
	% Second section with content
	\section{Second Section}
	\frame{...}
	% ...
\end{document}
	\end{lstlisting}
	
	\vfill
	
	\hfill \dots you need to compile twice to make the section work!
}

\end{document}
