\documentclass[11pt,compress,t]{beamer}
\usetheme{Boadilla}
\useinnertheme{circles}
\useoutertheme{shadow}
\usecolortheme{seahorse}
\usefonttheme{serif}
\setbeamertemplate{navigation symbols}{}

\setbeamercolor{myfootline}{bg=white,fg=blue}
\definecolor{myfoot}{rgb}{0.5,0.2,0.5}
\definecolor{darkblue}{rgb}{0.1,0,0.85}
\setbeamertemplate{headline}
  { \leavevmode\begin{beamercolorbox}[wd=\paperwidth,ht=1.25ex,dp=1ex,left]{}
    \end{beamercolorbox}}
\setbeamertemplate{footline}% 自定义页脚
  { \leavevmode\mbox{%
    \begin{beamercolorbox}[wd=.75\paperwidth,ht=2.25ex,dp=1ex,left]{myfootline}%
        \rule{2em}{0pt}\color{myfoot}\ttfamily\scriptsize%
        \insertshortauthor~(\insertshortinstitute)
    \end{beamercolorbox}%
    \begin{beamercolorbox}[wd=.25\paperwidth,ht=2.25ex,dp=1ex,right]{myfootline}%
       {\color{myfoot}\ttfamily\scriptsize\insertframenumber{}/%
        \inserttotalframenumber\hspace*{3ex}}
    \end{beamercolorbox}}
    \vskip0pt }

\setbeamercolor{frametitle}{fg=darkblue,bg=white}
\setbeamertemplate{frametitle}{%
  \leavevmode\linespread{1}\large\textbf{\insertframetitle}\par
  \color{green!50}\rule[8pt]{\linewidth}{2pt}\par\vspace{-1.2em}}

\setbeamertemplate{blocks}[default]
\setbeamersize{text margin left=0.75cm, text margin right=0.75cm}

\setbeamercolor{bluebox}{fg=black,bg=blue!10}
\setbeamercolor{redbox}{fg=black,bg=red!10}
\newenvironment{Boxblue}[1][\textwidth]
  {\begin{beamercolorbox}[sep=0.1em,shadow=true,wd=#1,rounded=true,center]{bluebox}}
  {\end{beamercolorbox}}
\newenvironment{Boxred}[1][\textwidth]
  {\begin{beamercolorbox}[sep=0.1em,shadow=true,wd=#1,rounded=true,center]{redbox}}
  {\end{beamercolorbox}}

\usepackage{amsmath,amssymb,amsfonts}
\usepackage{graphicx,xcolor}
\usepackage{hyperref}
\hypersetup{pdfborder=001,colorlinks=true,linkcolor=darkblue,urlcolor=blue}
\usepackage{bbding}
\newcommand{\Bullet}{{\fontsize{6pt}{6pt}\selectfont\CircleSolid}}
\newcommand{\Hand}{{\fontsize{8pt}{6pt}\selectfont\HandRight}}
\newcommand{\zhu}{{\color{blue!40}\Bullet}}
\newcommand{\zhuu}{{\color{red!80}\Hand}}
\newcommand{\labeli}{\zhu}
\newenvironment{myitem}
  {\begin{list}{{\hfill\raisebox{0pt}{\labeli}}}{%
    \setlength{\leftmargin}{1.2em}\labelwidth0.8em\labelsep.4em%
    \itemsep0pt\parsep2pt\itemindent0pt\topsep0pt}}{\end{list}}

\usepackage[framemethod=tikz]{mdframed}
\newmdenv[linecolor=green,middlelinewidth=0.5pt,%
          %outerlinewidth=0.5pt,skipabove=0pt,
          roundcorner=3pt,backgroundcolor=white,%
          innerbottommargin=3pt,innerrightmargin=5pt,%
          innerleftmargin=5pt,leftmargin=0ex]{mathbox}

%%%%%%%%%%%%%%%%%%%%%%%%%%%%%%%%%%%%%%%%%%%%%%%%%%%%%%%%%%%%%%%%%%%%%%%%%%%%%%
\renewcommand{\thefootnote}{}% 不要编号
\setbeamertemplate{footnote}{% 首行不缩进
  \noindent\insertfootnotemark%
  \scriptsize\color{blue!85!green!85}\insertfootnotetext\par\kern1ex}
\renewcommand\footnoterule%    更改横线属性:长度,粗细,颜色
  {\color{red}\kern-3pt\rule{0.4\linewidth}{0.5pt}\par\kern2.6pt}
%%%%%%%%%%%%%%%%%%%%%%%%%%%%%%%%%%%%%%%%%%%%%%%%%%%%%%%%%%%%%%%%%%%%%%%%%%%%%%

\renewcommand{\baselinestretch}{1.1}
%===== user defined commands =============================================
\newcommand{\bbm}{\begin{bmatrix}}
\newcommand{\ebm}{\end{bmatrix}}
\renewcommand{\C}{\mathbb{C}}
\newcommand{\R}{\mathbb{R}}
\newcommand{\lam}{\lambda}
\newcommand{\eps}{\varepsilon}
\newcommand{\dis}{\displaystyle}
\newcommand{\mycite}[1]{\textcolor{red}{\textrm{[#1]}}}
\newcommand{\diag}{\mathrm{diag}}


\begin{document}

\title{\bfseries\Large %
  This is title This is title\\[1ex]
   This is title This is title}
% \subtitle{subtitle if necessary}

\author[Author]%
  {Author\\ \url{email@email.edu.cn} }

\institute[Dept.Math]%
  {\small Department of xxxxx\\
          xxx xxx xxx University}

\date{{\small joint work with xxxx} \\[22pt]
              xxxx, xxxx, August 20xx}

% ===== title page =====
\begin{frame}[plain]
  \titlepage
\end{frame}

% ===== contents =====
\begin{frame}
  \frametitle{Outline}
   \tableofcontents[hideallsubsections] %[pausesections]
\end{frame}


% ===== main part =====
\section[Problem]{The Problem}
\begin{frame}
  \frametitle{The Problem}

  The weighted Toeplitz least square (WTLS) problem
  $$
     \min_{x\in\R^n} \| \Xi(Kx-f)\|_2^2 + u\|x\|_2^2
  $$
  \begin{myitem}
    \item[] - $\Xi \in\R^{m\times m}$ is a weighting matrix
        (usually positive diagonal)
    \item[] - $K\in\R^{m\times n}$ ($m\geq n$) is a full-rank
       Toeplitz related matrix
    \item[] - $\mu>0$ is a regularization parameter
    \item[] - $f$ is a given right-hand side
  \end{myitem}
  \medskip

  \zhuu\ Applications lead to WTLS
  \begin{myitem}
    \item[] - image reconstruction
    \item[] - image restoration with colored noise
    \item[] - nonlinear image restoration
  \end{myitem}

\end{frame}

\begin{frame}
  \frametitle{Applications lead to WTLS}
  \begin{myitem}
    \item Nonlinear image restoration
      $$ f = s(Kx) + \eta, $$
      where $f$, $x$, and $\eta$ represent the observed,
      the original image, and the noise vectors, respectively.

    \begin{myitem}
      \item[-] $s(\nu)$ denotes a point nonlinearity
      \item[-] $K$ is a blurring matrix \mycite{NCT99}
      \item[] - BTTB : Dirichlet boundary condition
      \item[] - BCCB : periodic boundary condition
      \item[] - BTHTHB : Neumann boundary condition
    \end{myitem}
  \end{myitem}

  \footnote{[NCT99]
    M. K. Ng, R. H. Chan and W.-C. Tang,
    \emph{A fast algorithm for deblurring models with Neumann boundary conditions},
    SIAM J. Sci. Comput., 21 (1999), 851--866.}

\end{frame}


\begin{frame}
  \frametitle{Equivalent Linear Systems -- Normal Equations}

  \begin{mathbox}\centering
    $\dis\min_{x\in\R^n} \| \Xi(Kx-f)\|_2^2 + u\|x\|_2^2 $
  \end{mathbox}
  \begin{myitem}
    \item Normal equations of WTLS
      $$ (K^T \Xi^T\Xi K + \mu I) x = K^T \Xi^T\Xi f $$

    \item[] \zhuu\ We can employ CG to solve this system
    \item[] \zhuu\ Disadvantages:
      \begin{myitem}
        \item[-] condition number is squared
        \item[-] CG may converge slow
        \item[-] well-suited preconditioner based on fast algorithms
                 is difficult to find
                  (because of the weighting matrix)
      \end{myitem}
  \end{myitem}
\end{frame}

\begin{frame}
  \frametitle{Equivalent Linear Systems -- Augmented System}
  \begin{myitem}
    \item Augmented system associated with WTLS problem
      $$
        \bbm W & K \\ K^T & -\mu I \ebm \bbm y\\x\ebm = \bbm f\\ 0\ebm,
      $$
      where $W = (\Xi^T \Xi)^{-1}$ and $y = \Xi^T \Xi(f - Kx)$

    \item[] \zhuu\ This is a generalized saddle point problem

    \item[] \zhuu\ Many solution methods are available
    \begin{myitem}
      \item[] - Uzawa, HSS, GSOR, ...
      \item[] - preconditioned Krylov subspace methods
    \end{myitem}
    \medskip

    \qquad
    \begin{Boxred}[0.7\textwidth]
      \textcolor{blue}{How to find a good preconditioner?}
    \end{Boxred}

  \end{myitem}
\end{frame}

% insert the contents at the beginning of every section.
\AtBeginSection[] % Do nothing for \section*
{ \begin{frame}<beamer> %\frametitle{Outline}
  \tableofcontents[currentsection,hideallsubsections]%,currentsubsection]
  \end{frame}
}

% ===== New frame =====
\section{HSS and AHSS Preconditioners}
\begin{frame}
  \frametitle{HSS Preconditioner \mycite{BN06}}

  Rewrite the augmented system into nonsymmetrix form
      $$
        \bbm W & K \\ -K^T & \mu I \ebm \bbm y\\x\ebm = \bbm f\\ 0\ebm
        \quad\text{or}\quad
        Au=c
      $$

  \begin{myitem}
    \item Hermitian and skew-Hermitian splitting \mycite{BGN03}
      $$ A=H+S $$
      where
      $$
        H = \bbm W & 0\\ 0 & \mu I\ebm \ \text{and}\
        S = \bbm 0 & K\\ -K^T & 0 \ebm
      $$

  \end{myitem}

  \footnote{[BN06]
    M. Benzi and M. K. Ng,
    \emph{Preconditioned iterative methods for weighted toeplitz
       least squares problems},
    SIAM J. Matrix Anal. Appl., 27 (2006), 1106--1124. }
  \footnote{[BGN03]
    Z.-Z. Bai, G. H. Golub and M. K. Ng,
    \emph{Hermitian and skew-Hermitian splitting methods for
    non-Hermitian positive definite linear systems},
    SIAM J. Matrix Anal. Appl., 24 (2003), 603--626. }

\end{frame}


\section{Numerical Experiments}
\begin{frame}
  \frametitle{Examples: One-Dimensional Problem \mycite{BN06}}

  \begin{mathbox}
  $K=[t_{ij}]\in\R^{n\times n}$ is a Toeplitz matrix defined by
  \begin{align*}
  \text{(i)}\quad
    & t_{ij}=\frac{1}{\sqrt{|i-j|}+1}
      &\to&\ \textcolor{blue}{well-conditioned} \\
  \text{(ii)}\quad
    & t_{ij}=\dfrac{1}{\sqrt{2\pi}\sigma}e^{\frac{-|i-j|^2}{2\sigma^2}}
     \ \text{with}\ \sigma=2
      &\to&\ \textcolor{blue}{ill-conditioned}
  \end{align*}
  \end{mathbox}

  Other parameters
  \begin{myitem}
    \item[] - $\Xi$: positive diagonal random matrix with
             $\kappa_2(\Xi)\approx 10^{3}$
    \item[] - $\mu=0.001$
    \item[] - stopping criterion:
      $
        \dfrac{\| c - Au\|_2}{\|c\|_2}<10^{-7}
      $\medskip
    \item[] -  Initial guess: zero vector
    \item[] -  maximum iteration steps: $1000$
  \end{myitem}
\end{frame}


% ===== New frame =====
\begin{frame}[plain]
\vspace{0.4\textheight}
\begin{center}
\Huge\color{red}\bfseries Thanks!
\end{center}
\end{frame}

\end{document}
