% ******************************************************************************
\section{Code}
% ******************************************************************************

% new slide
\frame{
  \frametitle{Code}
       \begin{itemize}
            \item <1-> This final section is just to show you how to add \textbf{code} and/or \textbf{equations} in
                        your presentations
        \end{itemize}
}

% new slide
\begin{frame}[t,fragile]{Introduction: Vectors}
    \begin{itemize}
        \item <1-> The basic structure to store data in R
        \item <2-> A vector is a one dimensional array $[a_1,a_2,...,a_n]$
        \item <3-> Common used function of "concatenate" in R: \textbf{c()}\\
        \pause
        \pause
          \vskip1ex
          \verb!x <- c(1,14,16,10)!\\
          \verb! [1] 1 14 16 10!\\
          \pause
          \verb!x <- c("hola","adios","hola de nuevo")!\\
          \verb! [1] "hola" "adios" "hola de nuevo"!\\
          \pause
          \verb!x <- c(TRUE,FALSE,TRUE)!\\
          \verb! [1] TRUE FALSE TRUE!
    \end{itemize}
\end{frame}


% new slide
\begin{frame}[fragile]
   \frametitle{Print code in LaTex using verbatim}
	\begin{verbatim}
for(i in 1:allGenes){
 if(expression[i] > threshold){
   expressed[i] <- gene.id[i]
 }
}
	\end{verbatim}
\end{frame}

% new slide
\begin{frame}[fragile]
   \frametitle{Print code in LaTex using lstlisting}
	\begin{lstlisting}
for(i in 1:allGenes){
 if(expression[i] > threshold){
   expressed[i] <- gene.id[i]
 }
}
	\end{lstlisting}
\end{frame}


% new slide
\frame{
  \frametitle{Print a demo equations}
    \begin{equation}
      x =\frac{-b\pm\sqrt{b^2-4ac}}{2a}
    \end{equation}
    \begin{equation}
      y= \sum_{i=0}^k i^k 
    \end{equation}    
}
