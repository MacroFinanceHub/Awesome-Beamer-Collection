%%
%%	This is file is the documentation for 'beamerthemeGYPT.sty'
%%
%%	The GYPT beamer theme is considered as a simple, neutral LaTeX beamer
%%	theme for use in GYPT reports.
%%	It uses the blue and green color of the GYPT logo.
%%
%%	THIS THEME IS NOT AN OFFICIAL ONE!!	
%%
%%	---------------------------------------------------------------------------
%%	Copyright 2017 Sebastian Friedl
%% 
%%	This work is licensed under a Creative Commons Attribution-ShareAlike 4.0
%%	International License (https://creativecommons.org/licenses/by-sa/4.0/).
%%
%%	This means that if you change the theme and re-distribute it, you must re-
%%	tain the copyright notice header and license it under the same CC-BY-SA
%%	license.
%%	This does not affect any presentations that you create with the theme.
%%
%%	---------------------------------------------------------------------------


% !TeX spellcheck = en_US

% !TeX document-id = {b3b4668f-f5d8-4010-ac4e-2eb3098d15f4}
% !TeX TXS-program:compile=txs:///pdflatex/[--shell-escape]

\documentclass[12pt,a4paper]{scrartcl}
\usepackage[utf8]{inputenc}

\usepackage[english]{babel}

\usepackage{amsmath}
\usepackage{amsfonts}
\usepackage{amssymb}
\usepackage{csquotes}
\usepackage{graphicx}
\usepackage{hyperref}
\usepackage{minted}
\usepackage{pgf}

\usepackage[osf,slantedGreek]{mathpazo}
\usepackage[osf,scale=.92]{roboto}

\parindent 0pt

\title{The GYPT \LaTeX\ beamer theme}
\author{Sebastian Friedl}
\date{March 24, 2017}

\hypersetup{pdftitle={The GYPT \LaTeX\ beamer theme},pdfauthor={Sebastian Friedl}}

\begin{document}
	\maketitle
	\thispagestyle{empty}
	
	
	\section*{Important note}
	This theme is an unofficial theme. \\
	It only uses the colors of the GYPT logo.
	
		
	\section*{Dependencies}
	The GYPT theme requires \LaTeXe\ and the following packages:
	\begin{itemize} \ttfamily
		\item appendixnumberbeamer
		\item etoolbox
		\item calc
		\item pgfopts
		\item keyval
		\item tikz
	\end{itemize}
	
	
	\section{Using the theme}
	For using the theme, you have to copy the file \texttt{beamerthemeGYPT.sty} into the folder containing the master file of your presentation. Advanced users may also install the style file on their local system. \par
	After that, you use the command \mintinline{LaTeX}{\usetheme{GYPT}} to set the theme used in your presentation to the GYPT theme.

	
	\section{Theme options}
	You can pass some options to the theme which influence how the theme looks. \\
	Syntax: \ \ \mintinline{LaTeX}{\usetheme[options]{GYPT}} \\\vspace{0.5ex}
	
	\textbf{Available options:}
	\begin{itemize}
		\item \texttt{numbering} \\
				Influences how the frames are numbered
				\begin{itemize}
					\item \texttt{none} \\
					The frames aren't numbered
					%
					\item \texttt{counter} \\
					The frames are numbered with a simple counter
					%
					\item \texttt{fraction}			\hfill \textit{default} \\
					The frames are numbered with current frame / total frames
					%
					\item \texttt{appedix} \\
					An \enquote{Appendix} remark is shown instead of the frame numbers. \\
					This setting gets automatically set for frames in the appendix.
				\end{itemize}
		%
		\item \texttt{footline} \\
				Influences how the footline of the theme looks
				\begin{itemize}
					\item \texttt{none} \\
					No footline is shown
					%
					\item \texttt{standard}			\hfill \textit{default} \\
					The standard footline containing author, title and framenumbers is shown.
					%
					\item \texttt{smallcaps} \\
					The standard footline with small caps font shape
				\end{itemize}
		%
		\item \texttt{titleformat title} \\
				Influences the style of the presentation title on the title frame
				\begin{itemize}
					\item \texttt{regular}			\hfill \textit{default}\\
					The presentation title is set in normal boldface style
					%
					\item \texttt{smallcaps} \\
					The presentation title is set in boldface and small caps style
				\end{itemize}
	\end{itemize}


	\section{Style sample}
	This style sample was made using the sample presentation \enquote{Steine} created by A.~Arzberger and S.~Friedl. \\
	It is licensed under the \textit{Creative Commons Attribution-ShareAlike 4.0 International} license.
	\begin{center}
		\pgfimage[width=0.4\textwidth,page=1]{steine_GYPT} \pgfimage[width=0.4\textwidth,page=2]{steine_GYPT} \\
		\pgfimage[width=0.4\textwidth,page=3]{steine_GYPT} \pgfimage[width=0.4\textwidth,page=4]{steine_GYPT} \\
		\pgfimage[width=0.4\textwidth,page=5]{steine_GYPT} \pgfimage[width=0.4\textwidth,page=6]{steine_GYPT} \\
		\pgfimage[width=0.4\textwidth,page=7]{steine_GYPT} \pgfimage[width=0.4\textwidth,page=8]{steine_GYPT}
	\end{center}

	
	\section{License}
	The GYPT theme is derived work from Matthias Vogelgesang's metropolis beamer theme. Thus, both copyright headers printed below apply.
	
	\subsection*{GYPT beamer theme}
	Copyright 2017 Sebastian Friedl \\
	This work is licensed under a \textit{Creative Commons Attribution-ShareAlike~4.0 International} License (\url{https://creativecommons.org/licenses/by-sa/4.0/}). \\
	This means that if you change the theme and re-distribute it, you must retain the copyright notice header and license it under the same CC-BY-SA license. \\
	This does not affect any presentations that you create with the theme.
	
	\subsection*{metropolis beamer theme}
	Copyright 2015 Matthias Vogelgesang and the LaTeX community. A full list of contributors can be found at
	\begin{center}
		\url{https://github.com/matze/mtheme/graphs/contributors}
	\end{center}
	and the original template was based on the HSRM theme by Benjamin Weiss.

	This work is licensed under a \textit{Creative Commons Attribution-ShareAlike~4.0 International} License (\url{https://creativecommons.org/licenses/by-sa/4.0/}).
\end{document}