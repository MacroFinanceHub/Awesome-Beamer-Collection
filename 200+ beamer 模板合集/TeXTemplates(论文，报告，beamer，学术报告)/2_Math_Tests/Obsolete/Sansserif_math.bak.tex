%%%%%%%%%%%%%%%%%%%%%%%%%%%%%%%%%%%%%%%%%%%%%%%%%
%%  SANS-SERIF MATH IN SANS-SERIF ENVIRONMENT  %%
%%%%%%%%%%%%%%%%%%%%%%%%%%%%%%%%%%%%%%%%%%%%%%%%%


\renewcommand{\rmdefault}{\savedrmdefaultformath}
\renewcommand{\seriesdefault}{\savedseriesdefaultformath}
\renewcommand{\familydefault}{\rmdefault}
\usepackage[subdued, nominus, LGRgreek, italic, itGreek]{mathastext}
% Even in "subdued" mode, mathastext replaces the minus sign by the en-dash. Reverse this:
%\DeclareMathSymbol{-}{\mathbin}{symbols}{0}

%\Mathastextstandardgreek
%\Mathastext[normal]
%\makeatletter
%\@for\@tempa:=a,b,c,d,e,f,g,h,i,j,k,l,m,n,o,p,q,r,s,t,u,v,w,x,y,z,%
%	A,B,C,D,E,F,G,H,I,J,K,L,M,N,O,P,Q,R,S,T,U,V,W,X,Y,Z%
%  	\do{\Mathastextsetmathskips{\@tempa}{2mu}{2mu}}%
%\makeatother
% \Mathastext[normal] destroys the Greek characters for mathdesign, restore them:
%\makeatletter
%\SetSymbolFont{letters}{normal}{OML}{\savedrmdefaultformath}{\savedseriesdefaultformath}{it}%
%\SetMathAlphabet{\mathbf}{normal}{T1}{\rmdefault}{\bfseries@rm}{it}
%\renewcommand{\seriesdefault}{\mdseries@sf}
%\makeatother
%\Mathastextcustomgreek
%\renewcommand{\familydefault}{\sfdefault}
%\Mathastextgreekfont{\familydefault}
%\Mathastext[sans]

% Redefine \mathit so that it also represents Greek letters correctly ==>
\def\italicizeWholeString #1{%
	\xItalicize#1$\wholeString%
}
\def\xItalicize#1#2\wholeString {%
	\if#1$%
	\else\say{#1}%
	\takeTheRest#2\ofTheString
	\fi%
}
\def\takeTheRest#1\ofTheString\fi {%
	\fi \xItalicize#1\wholeString%
}
\def\say#1{%
	\textit{$#1$\hspace{-0.05em}}%
}
\let \mathitorig \mathit
\renewcommand{\mathit}[1]{\italicizeWholeString{#1}}
% <==

\newcommand{\mathup}[1]{\mathrm{#1}}
\renewcommand{\mathbf}[1]{\bm{\mathrm{#1}}}
\newcommand{\mathbfit}[1]{\bm{\mathit{#1}}}

\Mathastextgreekfont{\sfdefault}
\renewcommand{\familydefault}{\sfdefault}

%\makeatletter
%\SetMathAlphabet{\mathbf}{normal}{T1}{\rmdefault}{\bfseries@rm}{it}
%\makeatother

% "sans" (\mdseries) mathversion ==>
\makeatletter
\renewcommand{\seriesdefault}{\mdseries@sf}
%\@for\@tempa:=a,b,c,d,e,f,g,h,i,j,k,l,m,n,o,p,q,r,s,t,u,v,w,x,y,z%
%	A,B,C,D,E,F,G,H,I,J,K,L,M,N,O,P,Q,R,S,T,U,V,W,X,Y,Z%
%  	\do{\Mathastextsetmathskips{\@tempa}{2mu}{2mu}}%
\makeatother
% We cannot use \MathastextDeclareVersion{...} for sans and sansbold, because this would apply either italics also to all digits and delimiters or upright to all variables.
\Mathastext[sans]
% Take symbols from MathDesign Charter in the sans mathversion,
% because there is no sans-serif font containing summation symbols etc.:
\SetSymbolFont{symbols}     {sans}{OMS}{mdbch}  {m}{n}%
\SetSymbolFont{largesymbols}{sans}{OMX}{mdbch}  {m}{n}%
\SetMathAlphabet{\mathcal}  {sans}{OMS}{xmdcmsy}{m}{n}%
%\DeclareSymbolFontAlphabet{\mathcal}{mdcal}
%\DeclareSymbolFont{mdscr}{OMS}{mdbch}{m}{n}%
\SetMathAlphabet{\mathfrak} {sans}{U}  {xmdeuf} {m}{n}%
%% From mathastext.sty:
%\DeclareMathSymbol{-}{\mathbin}{mtoperatorfont}{21}
%% <==

% "sansbold" mathversion ==>
\makeatletter
\renewcommand{\seriesdefault}{\bfseries@sf}
\makeatother
\Mathastext[sansbold]
% Take symbols from MathDesign Charter in the sansbold mathversion,
% because there is no sans-serif font containing summation symbols etc.:
\SetSymbolFont{symbols}     {sansbold}{OMS}{mdbch}  {b}{n}%
\SetSymbolFont{largesymbols}{sansbold}{OMX}{mdbch}  {b}{n}%
\SetMathAlphabet{\mathcal}  {sansbold}{OMS}{xmdcmsy}{b}{n}%
\SetMathAlphabet{\mathfrak} {sansbold}{U}  {xmdeuf} {b}{n}%
% <==
\Mathastextupgreek
\MathastextupGreek
\makeatletter
\MathastextDeclareVersion[n]{sansup}    {T1}{\familydefault}{\mdseries@sf}{n}
\MathastextDeclareVersion[n]{sansupbold}{T1}{\familydefault}{\bfseries@sf}{n}
\makeatother

% Loading siunitx eats up math alphabets in the "normal" mathversion.
% Avoid the "Too many math alphabets used in verison normal" error by
% relegating \mathcal and \mathfrak to a separate version. ==>
\MathastextDeclareVersion[n]{calligraphic}{OMS}{xmdcmsy}{m}{n}
\SetMathAlphabet{\mathcal}  {calligraphic}{OMS}{xmdcmsy}{m}{n}%
\SetMathAlphabet{\mathfrak} {calligraphic}{U}  {xmdeuf} {m}{n}
\let \mathcalaux  \mathcal
\let \mathfrakaux \mathfrak
\renewcommand{\mathcal}[1]{%
	\textnormal{\IfInSansMode$\mathcalaux{#1}$%
	\else\mathastextversion{calligraphic}$\mathcalaux{#1}$\mathversion{normal}%
	\fi\relax}%
}
\renewcommand{\mathfrak}[1]{%
	\textnormal{\IfInSansMode$\mathfrakaux{#1}$%
	\else\mathastextversion{calligraphic}$\mathfrakaux{#1}$\mathversion{normal}%
	\fi\relax}%
}
% <==

% Take additional math symbols from the sans-serif family ==>
\let \oldpm    \pm
\let \oldtimes \times
\let \olddiv   \div
\makeatletter
\newcommand{\pmsf}   {\mathbin{\textnormal{\usefont{TS1}{\sfdefault}{\f@series}{n}\char"B1}}}
\newcommand{\timessf}{\mathbin{\textnormal{\usefont{TS1}{\sfdefault}{\f@series}{n}\char"D6}}}
\newcommand{\divsf}  {\mathbin{\textnormal{\usefont{TS1}{\sfdefault}{\f@series}{n}\char"F6}}}
\makeatother
% <==

\newif\IfInSansMode
\newif\IfInBoldMode
\let \mathrmaux \mathrm
\newcommand{\renewmathcommands}{%
	\renewcommand{\mathnormal}[1]{\textsf{$##1$}}%
	\renewcommand{\mathrm}[1]{%
		\textrm{\mathversion{normal}\Mathastextstandardgreek$\mathrmaux{##1}$\mathastextversion{sans}}%
	}%
	\renewcommand{\mathup}[1]{%
		\textsf{\IfInBoldMode\mathastextversion{sansupbold}$##1$\mathastextversion{sansbold}\else\mathastextversion{sansup}$##1$\mathastextversion{sans}\fi\relax}%
	}%
	\renewcommand{\mathit}  [1]{\textsf{\mathastextversion{sans}$##1$}}%
	\renewcommand{\mathbf}  [1]{\textsf{\mathastextversion{sansbold}$##1$\mathastextversion{sans}}}%
	\renewcommand{\mathbfit}[1]{\textnormal{\mathastextversion{sansbold}$##1$\mathastextversion{sans}}}%
	\renewcommand{\pm}   {\pmsf}%
	\renewcommand{\times}{\timessf}%
	\renewcommand{\div}  {\divsf}%
}

\newcommand{\sansmath}{%
	\InSansModetrue%
	\mathastextversion{sans}%
	\renewmathcommands%
}
\newcommand{\unsansmath}{%
	\mathversion{normal}%
	\InSansModefalse%
}
% In subdued mode, the "normal" math version is NOT subject to \Mathastext

\appto{\sffamily}{\sansmath}
\appto{\mdseries}{\InBoldModefalse\IfInSansMode\sansmath\fi\relax}
\appto{\bfseries}{\InBoldModetrue\IfInSansMode\mathastextversion{sansbold}\renewmathcommands\else\mathversion{bold}\fi\relax}
\appto{\normalfont}{\unsansmath}
\appto{\rmfamily}{\unsansmath}

% Revert to the initial font families ==>
\renewcommand{\rmdefault}{\savedrmdefaultfortext}
\renewcommand{\seriesdefault}{\savedseriesdefaultfortext}
\renewcommand{\familydefault}{\rmdefault}
% <==

%%%%%%%%%%%%%%%%%%%%%%%%%%%%%%%%%%%%%%%%%%%%%%%%%

% Fill in ``missing'' Greek glyphs for completeness (not really necessary,
% since they look identical to Latin glyphs and are thus almost never used) ==>
\newcommand{\omicron}{o}
\newcommand{\Digamma}{F}
\newcommand{\Alpha}  {A}
\newcommand{\Beta}   {B}
\newcommand{\Epsilon}{E}
\newcommand{\Zeta}   {Z}
\newcommand{\Eta}    {H}
\newcommand{\Iota}   {I}
\newcommand{\Kappa}  {K}
\newcommand{\Mu}     {M}
\newcommand{\Nu}     {N}
\newcommand{\Omicron}{O}
\newcommand{\Rho}    {P}
\newcommand{\Tau}    {T}
\newcommand{\Chi}    {X}
% <==