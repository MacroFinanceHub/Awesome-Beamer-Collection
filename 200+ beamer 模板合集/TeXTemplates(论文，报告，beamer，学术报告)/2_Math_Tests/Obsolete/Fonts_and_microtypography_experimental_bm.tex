% !TeX TXS-program:compile = txs:///pdflatex/
% !TeX TS-program = pdflatex
% !TeX TXS-program:bibliography = txs:///biber
% !BIB program = biber



%%%%%%%%%%%%%%%%%%%%%%%%%%%%%%%%%%%%
%%  FONT SETTINGS AND TYPOGRAPHY  %%
%%%%%%%%%%%%%%%%%%%%%%%%%%%%%%%%%%%%


\usepackage[LGR, T1]{fontenc}

\usepackage[charter, uppercase=italicized, greeklowercase=italicized, greekuppercase=italicized, sfscaled=false, ttscaled=false]{mathdesign}	% Use Bitstream Charter as the math font
\usepackage[scaled=.96, lining, sups]{XCharter}	% Use Bitstream Charter as the text font
%\renewcommand{\rmdefault}{XCharter-TLF}

\def\hyphenateWholeString #1{\xHyphenate#1$\wholeString}

\def\xHyphenate#1#2\wholeString {\if#1$%
\else\say{#1}%
\takeTheRest#2\ofTheString
\fi}

\def\takeTheRest#1\ofTheString\fi
{\fi \xHyphenate#1\wholeString}

\def\say#1{\textit{$#1$\hspace{-0.075em}}}

\let \mathitorig \mathit
\renewcommand{\mathit}[1]{\hyphenateWholeString{#1}}

\newcommand{\bmmax}{1}  
\newcommand{\hmmax}{0}  
\usepackage{bm}

\newcommand{\mathup}[1]{\mathrm{#1}}
\renewcommand{\mathbf}[1]{\bm{\mathrm{#1}}}
\newcommand{\mathbfit}[1]{\bm{\mathit{#1}}}


%\DeclareSymbolFont{rmcustom}{T1}{\rmdefault}{\mddefault}{\updefault}

% \DeclareMathSymbol {<symbol>} {<type>} {<sym-font>} {<slot>}
% Type              Meaning	            Example
% 0 or \mathord     Ordinary             $\alpha$
% 1 or \mathop      Large operator       $\sum$
% 2 or \mathbin     Binary operation     $\times$
% 3 or \mathrel     Relation             $\leq$
% 4 or \mathopen    Opening              $\langle$
% 5 or \mathclose   Closing              $\rangle$
% 6 or \mathpunct   Punctuation          ;
% 7 or \mathalpha   Alphabet character   A
% Example declaration:
% \DeclareMathSymbol{b}{3}{letters}{`b}

%\DeclareMathSymbol{(}{\mathopen}{rmcustom}{`(}
%\DeclareMathSymbol{)}{\mathclose}{rmcustom}{`)}
%\DeclareMathSymbol{[}{\mathopen}{rmcustom}{`[}
%\DeclareMathSymbol{]}{\mathclose}{rmcustom}{`]}
%\DeclareMathSymbol{/}{\mathop}{rmcustom}{`/}	% Increase spacing around slash/solidus
%\DeclareMathDelimiter{\lbrace}{\mathopen}{rmcustom}{`\{}{largesymbols}{"08}
%\DeclareMathDelimiter{\rbrace}{\mathclose}{rmcustom}{`\}}{largesymbols}{"09}

% Reduce/prevent stretching of spaces around operators in inline formulae ==>
% See http://tex.stackexchange.com/questions/83746/keeping-the-distance-between-mathematical-symbols-consistent
% \thinmuskip=3.0mu (already without glue)
\medmuskip=1\medmuskip	% Formerly 4.0mu plus 2.0mu minus 4.0mu -> 4.0mu
\thickmuskip=1\thickmuskip	% Formerly 5.0mu plus 5.0mu -> 5.0mu
% <==
\spaceskip=\fontdimen2\font plus \fontdimen3\font minus 0.75\fontdimen4\font	% Reduce by how much an interword space can shrink
% See http://tex.stackexchange.com/questions/19236/how-to-change-the-interword-spacing and http://tex.stackexchange.com/questions/88991/what-do-different-fontdimennum-mean

% Increase spacing around relational symbols (<, =, >, \le, \ge, \equiv; +, -, \times, etc.) in display formulae ==>
\newcommand{\thickmu}{\thickmuskip=12mu \medmuskip=6mu}
\AtBeginEnvironment{equation}{\thickmu}
\AtBeginEnvironment{equation*}{\thickmu}
\AtBeginEnvironment{align}{\thickmu}
\AtBeginEnvironment{align*}{\thickmu}
\AtBeginEnvironment{eqnarray}{\thickmu}
\AtBeginEnvironment{eqnarray*}{\thickmu}
% <==

\usepackage[book, semibold, lining, tabular, scale=0.91]{FiraSans}[2018/05/23]
\makeatletter
\@ifpackagelater{FiraSans}{2018/05/23}
	{}
	{\PackageError{FiraSans}
	     {Outdated 'FiraSans' package}
	     {Upgrade to FiraSans 2018/05/23 or newer!\MessageBreak
	      Otherwise the sans-serif font may not work properly, or the compilation might fail.\MessageBreak
	      This is a fatal error. I'm aborting now.}%
	   \endinput}
\makeatother
%% ``Medium'' as the bold series:
%\makeatletter
%\def\bfseries@sf{mb}
%\makeatother
%% In case you wish to use the ``light'' options of FiraSans: -->
%\usepackage{afterpackage}
%\AfterPackage{siunitx}{%
%  \providecommand*{\lseries}{\fontseries{l}\selectfont}
%}
%% <--
%% See https://tex.stackexchange.com/questions/164791/combination-of-siunitx-and-light-weight-font-fails-to-compile.
\usepackage[lining, scale=0.91]{FiraMono}


% The microtype package enables so-called hanging punctuation. That is, when a punctuation sign like ":", ".", "-", etc. is found at the beginning or end of a line, it is protruded a little into the page margin. This results in "optical margin alignment," because the protrusion makes the margin alignment look straighter.
\ifxetex
	\usepackage[protrusion=true, expansion=false]{microtype}
	% Apart from this, do nothing, since \DisableLigatures[f]{family = {rm*}} and \SetExtraKerning are not available in Xe(La)TeX.
\else
	\usepackage[protrusion=true, expansion=false, kerning=true]{microtype}
	\DisableLigatures[f]{family = {rm*}}
	% Disable the f* ligatures for Charter because the font does not provide sufficient support.
	\SetExtraKerning[unit=space]
	{encoding=*, family=*, series=*, size={*, normalsize, footnotesize}, font = */*/*/*/*}
	{\textemdash = {325, 325},
		/ = {100, 100},
		: = { 50,   0},
		; = { 50,   0}}
\fi

\frenchspacing	% Prevent increased whitespace after periods
\sloppy

\usepackage{xfrac}	% Provides \sfrac



%%%%%%%%%%%%%%%%%%%%%%%%%%%%%%%%%%%%%%%%%%%%%%%%%
%%  SANS-SERIF MATH IN SANS-SERIF ENVIRONMENT  %%
%%%%%%%%%%%%%%%%%%%%%%%%%%%%%%%%%%%%%%%%%%%%%%%%%


\makeatletter
\SetMathAlphabet{\mathbf}{normal}{T1}{\rmdefault}{\bfseries@rm}{it}
\renewcommand{\seriesdefault}{\mdseries@sf}
\makeatother
\renewcommand{\familydefault}{\sfdefault}
\usepackage[LGRgreek, subdued, italic, itGreek]{mathastext}
\makeatletter
\@for\@tempa:=a,b,c,d,e,f,g,h,i,j,k,l,m,n,o,p,q,r,s,t,u,v,w,x,y,z%
	A,B,C,D,E,F,G,H,I,J,K,L,M,N,O,P,Q,R,S,T,U,V,W,X,Y,Z%
  	\do{\MTsetmathskips{\@tempa}{2mu}{2mu}}%
\makeatother
\Mathastext[sans]
\makeatletter
\MTDeclareVersion[it]{mathbfit}{T1}{mdbch}{b}{it}
\renewcommand{\seriesdefault}{\bfseries@sf}
\makeatother
\Mathastext[sansbold]
\makeatletter
\renewcommand{\seriesdefault}{\mdseries@sf}
\makeatother
\Mathastextupgreek
\MathastextupGreek
%\renewcommand{\Mathastextshape}{\upshape}
\makeatletter
\MTDeclareVersion[n]{sansup}{T1}{\familydefault}{\mdseries@sf}{n}
\MTDeclareVersion[n]{sansupbold}{T1}{\familydefault}{\bfseries@sf}{n}
\makeatother
\renewcommand{\familydefault}{\rmdefault}
\makeatletter
\renewcommand{\seriesdefault}{\mdseries@rm}
\makeatother
%% Greek letters in the sans-serif font in normal, italic, bold, and bold italic -->
%%\DeclareSymbolFont{sansgreeklettersup}        {LGR}{\sfdefault}{\mdseries@sf}{n}
%\DeclareSymbolFont{sansgreeklettersit}        {LGR}{\sfdefault}{\mdseries@sf}{it}
%%\DeclareSymbolFont{sansgreeklettersbfup}      {LGR}{\sfdefault}{\mdseries@sf}{n}
%%\DeclareSymbolFont{sansgreeklettersbfit}      {LGR}{\sfdefault}{\mdseries@sf}{it}
%% <--
%\SetSymbolFont{sansgreeklettersit}  {sans}    {LGR}{\sfdefault}{\mdseries@sf}{it}
%\SetSymbolFont{sansgreeklettersit}  {sansbold}{LGR}{\sfdefault}{\bfseries@sf}{it}
%% The symbol fonts need to be set for BOTH sans math versions so that one can use \mathnormal, \mathit, \mathbf, etc. in BOTH the sans and sansbold math versions. -->
%%\SetSymbolFont{sansgreeklettersup}  {sans}    {LGR}{\sfdefault}{\mdseries@sf}{n}
%%\SetSymbolFont{sansgreeklettersit}  {sans}    {LGR}{\sfdefault}{\mdseries@sf}{it}
%%\SetSymbolFont{sansgreeklettersbfup}{sans}    {LGR}{\sfdefault}{\bfseries@sf}{n}
%%\SetSymbolFont{sansgreeklettersbfit}{sans}    {LGR}{\sfdefault}{\bfseries@sf}{it}
%%\SetSymbolFont{sansgreeklettersup}  {sansbold}{LGR}{\sfdefault}{\mdseries@sf}{n}
%%\SetSymbolFont{sansgreeklettersit}  {sansbold}{LGR}{\sfdefault}{\mdseries@sf}{it}
%%\SetSymbolFont{sansgreeklettersbfup}{sansbold}{LGR}{\sfdefault}{\bfseries@sf}{n}
%%\SetSymbolFont{sansgreeklettersbfit}{sansbold}{LGR}{\sfdefault}{\bfseries@sf}{it}
%% <--
%%% To use a specific font, e.g., DejaVu Sans -->
%%\DeclareSymbolFont{sansgreeklettersit}{LGR}{DejaVuSans-TLF}{m}{it}%  for greek letters
%%\SetSymbolFont{sansgreeklettersit}{sans}{LGR}{DejaVuSans-TLF}{m}{it}%  for greek letters
%%\SetSymbolFont{sansgreeklettersit}{sansbold}{LGR}{DejaVuSans-TLF}{b}{it}%  for greek letters
%%% <--
%%\SetMathAlphabet{\mathit}{sans}{T1}{\sfdefault}{\mdseries@sf}{it}
%%\SetMathAlphabet{\mathit}{sansbold}{T1}{\sfdefault}{\bfseries@sf}{it}
%\makeatother
%% Take symbols from MathDesign Charter in sans mathversion -->
%\SetSymbolFont{symbols}     {sans}{OMS}{mdbch}{m}{n}%
%\SetSymbolFont{largesymbols}{sans}{OMX}{mdbch}{m}{n}%
%\SetSymbolFont{symbols}     {sansbold}{OMS}{mdbch}{b}{n}%
%\SetSymbolFont{largesymbols}{sansbold}{OMX}{mdbch}{b}{n}%
%% <--
%
%% The following is taken from
%% https://tex.stackexchange.com/questions/116389/automatic-upright-math-when-text-is-in-italic/116399#116399
%
%% Filling in ``missing'' Greek glyphs for completeness
%% (not really necessary, since they look identical to Latin glyphs and are thus almost never used)
%\newcommand{\omicron}{o}
%\newcommand{\Digamma}{F}
%\newcommand{\Alpha}  {A}
%\newcommand{\Beta}   {B}
%\newcommand{\Epsilon}{E}
%\newcommand{\Zeta}   {Z}
%\newcommand{\Eta}    {H}
%\newcommand{\Iota}   {I}
%\newcommand{\Kappa}  {K}
%\newcommand{\Mu}     {M}
%\newcommand{\Nu}     {N}
%\newcommand{\Omicron}{O}
%\newcommand{\Rho}    {P}
%\newcommand{\Tau}    {T}
%\newcommand{\Chi}    {X}
%% Save original definitions of the Greek letters -->
%\makeatletter
%\@for\@tempa:=%
%alpha,beta,gamma,delta,epsilon,zeta,eta,theta,iota,kappa,lambda,mu,nu,xi,%
%omicron,pi,rho,sigma,varsigma,tau,upsilon,phi,chi,psi,omega,digamma,%
%Alpha,Beta,Gamma,Delta,Epsilon,Zeta,Eta,Theta,Iota,Kappa,Lambda,Mu,Nu,Xi,%
%Omicron,Pi,Rho,Sigma,Tau,Upsilon,Phi,Chi,Psi,Omega,Digamma%
%\do{\expandafter\let\csname\@tempa orig\expandafter\endcsname\csname\@tempa\endcsname}%
%\makeatother
%% <--
%
%%% 'uppercase'
%\DeclareMathSymbol{\AlphaLGR}   {\mathalpha}{sansgreeklettersit}{`A}
%\DeclareMathSymbol{\BetaLGR}    {\mathalpha}{sansgreeklettersit}{`B}
%\DeclareMathSymbol{\GammaLGR}   {\mathalpha}{sansgreeklettersit}{`G}
%\DeclareMathSymbol{\DeltaLGR}   {\mathalpha}{sansgreeklettersit}{`D}
%\DeclareMathSymbol{\EpsilonLGR} {\mathalpha}{sansgreeklettersit}{`E}
%\DeclareMathSymbol{\ZetaLGR}    {\mathalpha}{sansgreeklettersit}{`Z}
%\DeclareMathSymbol{\EtaLGR}     {\mathalpha}{sansgreeklettersit}{`H}
%\DeclareMathSymbol{\ThetaLGR}   {\mathalpha}{sansgreeklettersit}{`J}
%\DeclareMathSymbol{\IotaLGR}    {\mathalpha}{sansgreeklettersit}{`I}
%\DeclareMathSymbol{\KappaLGR}   {\mathalpha}{sansgreeklettersit}{`K}
%\DeclareMathSymbol{\LambdaLGR}  {\mathalpha}{sansgreeklettersit}{`L}
%\DeclareMathSymbol{\MuLGR}      {\mathalpha}{sansgreeklettersit}{`M}
%\DeclareMathSymbol{\NuLGR}      {\mathalpha}{sansgreeklettersit}{`N}
%\DeclareMathSymbol{\XiLGR}      {\mathalpha}{sansgreeklettersit}{`X}
%\DeclareMathSymbol{\OmicronLGR} {\mathalpha}{sansgreeklettersit}{`O}
%\DeclareMathSymbol{\PiLGR}      {\mathalpha}{sansgreeklettersit}{`P}
%\DeclareMathSymbol{\RhoLGR}     {\mathalpha}{sansgreeklettersit}{`R}
%\DeclareMathSymbol{\SigmaLGR}   {\mathalpha}{sansgreeklettersit}{`S}
%\DeclareMathSymbol{\TauLGR}     {\mathalpha}{sansgreeklettersit}{`T}
%\DeclareMathSymbol{\UpsilonLGR} {\mathalpha}{sansgreeklettersit}{`U}
%\DeclareMathSymbol{\PhiLGR}     {\mathalpha}{sansgreeklettersit}{`F}
%\DeclareMathSymbol{\ChiLGR}     {\mathalpha}{sansgreeklettersit}{`Q}
%\DeclareMathSymbol{\PsiLGR}     {\mathalpha}{sansgreeklettersit}{`Y}
%\DeclareMathSymbol{\OmegaLGR}   {\mathalpha}{sansgreeklettersit}{`W}
%\DeclareMathSymbol{\DigammaLGR} {\mathalpha}{sansgreeklettersit}{195}
%%% 'lowercase'
%\DeclareMathSymbol{\alphaLGR}   {\mathalpha}{sansgreeklettersit}{`a}
%\DeclareMathSymbol{\betaLGR}    {\mathalpha}{sansgreeklettersit}{`b}
%\DeclareMathSymbol{\gammaLGR}   {\mathalpha}{sansgreeklettersit}{`g}
%\DeclareMathSymbol{\deltaLGR}   {\mathalpha}{sansgreeklettersit}{`d}
%\DeclareMathSymbol{\epsilonLGR} {\mathalpha}{sansgreeklettersit}{`e}
%\DeclareMathSymbol{\zetaLGR}    {\mathalpha}{sansgreeklettersit}{`z}
%\DeclareMathSymbol{\etaLGR}     {\mathalpha}{sansgreeklettersit}{`h}
%\DeclareMathSymbol{\thetaLGR}   {\mathalpha}{sansgreeklettersit}{`j}
%\DeclareMathSymbol{\iotaLGR}    {\mathalpha}{sansgreeklettersit}{`i}
%\DeclareMathSymbol{\kappaLGR}   {\mathalpha}{sansgreeklettersit}{`k}
%\DeclareMathSymbol{\lambdaLGR}  {\mathalpha}{sansgreeklettersit}{`l}
%\DeclareMathSymbol{\muLGR}      {\mathalpha}{sansgreeklettersit}{`m}
%\DeclareMathSymbol{\nuLGR}      {\mathalpha}{sansgreeklettersit}{`n}
%\DeclareMathSymbol{\xiLGR}      {\mathalpha}{sansgreeklettersit}{`x}
%\DeclareMathSymbol{\omicronLGR} {\mathalpha}{sansgreeklettersit}{`o}
%\DeclareMathSymbol{\piLGR}      {\mathalpha}{sansgreeklettersit}{`p}
%\DeclareMathSymbol{\rhoLGR}     {\mathalpha}{sansgreeklettersit}{`r}
%\DeclareMathSymbol{\sigmaLGR}   {\mathalpha}{sansgreeklettersit}{`s}
%\DeclareMathSymbol{\varsigmaLGR}{\mathalpha}{sansgreeklettersit}{`c}
%\DeclareMathSymbol{\tauLGR}     {\mathalpha}{sansgreeklettersit}{`t}
%\DeclareMathSymbol{\upsilonLGR} {\mathalpha}{sansgreeklettersit}{`u}
%\DeclareMathSymbol{\phiLGR}     {\mathalpha}{sansgreeklettersit}{`f}
%\DeclareMathSymbol{\chiLGR}     {\mathalpha}{sansgreeklettersit}{`q}
%\DeclareMathSymbol{\psiLGR}     {\mathalpha}{sansgreeklettersit}{`y}
%\DeclareMathSymbol{\omegaLGR}   {\mathalpha}{sansgreeklettersit}{`w}
%\DeclareMathSymbol{\digammaLGR} {\mathalpha}{sansgreeklettersit}{147}
%
% Based on description of the TS1 encoding in
% http://ctan.math.illinois.edu/macros/latex/doc/encguide.pdf:
\newcommand{\mathrmaux}[1]{\mathrm{#1}}
%\renewcommand{\mathbfit}[1]{\text{\mathastextversion{mathbfit}$#1$\mathastextversion{normal}}}
\newcommand{\renewmathcommands}{%
	\renewcommand{\mathnormal}[1]{\textsf{$##1$}}%
	\renewcommand{\mathrm}    [1]{\text{\mathastextversion{normal}$\mathrmaux{##1}$\mathastextversion{sans}}}%
	\renewcommand{\mathup}    [1]{\text{\IfInBoldMode\mathastextversion{sansupbold}$##1$\mathastextversion{sansbold}\else\mathastextversion{sansup}$##1$\mathastextversion{sans}\fi\relax}}%
	\renewcommand{\mathit}  [1]{\textit{$##1$}}%
	\renewcommand{\mathbf}  [1]{\textbf{$##1$}}%
	\renewcommand{\mathbfit}[1]{\text{\mathastextversion{sansbold}$##1$\mathastextversion{sans}}}%
}
\let \oldpm    \pm
\let \oldtimes \times
\let \olddiv   \div
\makeatletter
\newcommand{\pmsf}   {\mathbin{\text{\usefont{TS1}{\sfdefault}{\f@series}{n}\char"B1}}}
\newcommand{\timessf}{\mathbin{\text{\usefont{TS1}{\sfdefault}{\f@series}{n}\char"D6}}}
\newcommand{\divsf}  {\mathbin{\text{\usefont{TS1}{\sfdefault}{\f@series}{n}\char"F6}}}
\newcommand*{\sansmath}{%
	\mathastextversion{sans}%
%	\@for\@tempa:=%
%	alpha,beta,gamma,delta,epsilon,zeta,eta,theta,iota,kappa,lambda,mu,nu,xi,%
%	omicron,pi,rho,sigma,varsigma,tau,upsilon,phi,chi,psi,omega,digamma,%
%	Alpha,Beta,Gamma,Delta,Epsilon,Zeta,Eta,Theta,Iota,Kappa,Lambda,Mu,Nu,Xi,%
%	Omicron,Pi,Rho,Sigma,Tau,Upsilon,Phi,Chi,Psi,Omega,Digamma%
%	\do{\expandafter\let\csname\@tempa\expandafter\endcsname\csname\@tempa LGR\endcsname}%
	\renewcommand{\pm}{\pmsf}%
	\renewcommand{\times}{\timessf}%
	\renewcommand{\div}{\divsf}%
	\renewmathcommands%
}
\newcommand*{\unsansmath}{%
	\mathversion{normal}
%	\@for\@tempa:=%
%	alpha,beta,gamma,delta,epsilon,zeta,eta,theta,iota,kappa,lambda,mu,nu,xi,%
%	omicron,pi,rho,sigma,varsigma,tau,upsilon,phi,chi,psi,omega,digamma,%
%	Alpha,Beta,Gamma,Delta,Epsilon,Zeta,Eta,Theta,Iota,Kappa,Lambda,Mu,Nu,Xi,%
%	Omicron,Pi,Rho,Sigma,Tau,Upsilon,Phi,Chi,Psi,Omega,Digamma%
%	\do{\expandafter\let\csname\@tempa\expandafter\endcsname\csname\@tempa orig\endcsname}%
	\renewcommand{\pm}{\oldpm}%
	\renewcommand{\times}{\oldtimes}%
	\renewcommand{\div}{\olddiv}%
}
\newcommand*{\checkgreekletters}{%
	\@for\@tempa:=%
	alpha,beta,gamma,delta,epsilon,zeta,eta,theta,iota,kappa,lambda,mu,nu,xi,%
	omicron,pi,rho,sigma,varsigma,tau,upsilon,phi,chi,psi,omega,digamma,%
	Alpha,Beta,Gamma,Delta,Epsilon,Zeta,Eta,Theta,Iota,Kappa,Lambda,Mu,Nu,Xi,%
	Omicron,Pi,Rho,Sigma,Tau,Upsilon,Phi,Chi,Psi,Omega,Digamma%
	\do{$\csname\@tempa\endcsname,$ }%
}
\makeatother

%
%\newif\IfInSansMode
%\appto{\sffamily}{\InSansModetrue\mathastextversion{sans}\renewmathcommands\sansmath}
%\appto{\mdseries}{\IfInSansMode\mathastextversion{sans}\fi\relax}
%\appto{\bfseries}{\IfInSansMode\mathastextversion{sansbold}\else\mathversion{bold}\fi\relax}
%\appto{\normalfont}{\unsansmath\mathversion{normal}\InSansModefalse}
%\appto{\rmfamily}{\unsansmath\mathversion{normal}\InSansModefalse}

\newif\IfInSansMode
\newif\IfInBoldMode
\appto{\sffamily}{\InSansModetrue\sansmath}
\appto{\mdseries}{\InBoldModefalse\IfInSansMode\sansmath\fi\relax}
\appto{\bfseries}{\InBoldModetrue\IfInSansMode\mathastextversion{sansbold}\renewmathcommands\else\mathversion{bold}\fi\relax}
\appto{\normalfont}{\unsansmath\InSansModefalse}
\appto{\rmfamily}{\unsansmath\InSansModefalse}
