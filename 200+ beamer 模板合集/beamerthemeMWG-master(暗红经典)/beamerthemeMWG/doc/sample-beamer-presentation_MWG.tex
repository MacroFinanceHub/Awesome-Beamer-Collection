%%	This is file 'sample-beamer-presentation.tex', Version 2017-08-03
%%
%%	The initial version has been written by Sebastian Friedl in 2017 <sfr682k@t-online.de>
%%	Visit the GitHub repository at https://github.com/SFr682k/sample-latex-beamer-presentation
%%	It doesn't harm anyone if you do some advertising ... :D
%% 
%%	THIS WORK IS LICENSED UNDER THE "DO WHAT THE FUCK YOU WANT TO PUBLIC LICENSE"
%%	PERMISSION IS GRANTED TO DO  E V E R Y T H I N G  WITH THIS WORK.
%%
%%	Just a note to authors of LaTeX beamer packages and themes:
%%	This file is nice to show your package's behavior, how your theme looks like, ... ;)
%%
%%	-------------------------------------------------------------------------------------------
%%
%%	This file is a sample LaTeX beamer presentation in English language pointing out the main
%%	advantages of using LaTeX beamer for creating your presentations.
%%
%%	Just use this presentation to see how different beamer themes look.	Change them in line 42.
%%	The default theme is the 'default' theme. :D
%%
%%	If the font size is too big for the theme, change it in line 33. Default is 11pt.
%%
%%	Maybe you would also like to remove or comment the separation frame generated in line 150.
%%
%%	Of course you are allowed to create whole frames :)
%%
%%	Have fun with TeXing ;)
%%
%%	-------------------------------------------------------------------------------------------


% YOU ARE ABLE TO CHANGE THE FONT SIZE HERE, IF NECESSARY
\documentclass[11pt]{beamer}

\usepackage[utf8]{inputenc}
\usepackage[T1]{fontenc}
\usepackage[english]{babel}
\usepackage{csquotes}
\usepackage{verbatim}

% SWITCH THE THEME USED FOR THIS PRESENTATION HERE:
\usetheme{MWG}

\usepackage[charter]{mathdesign}
\usepackage[osf]{XCharter}
\usepackage[osf,scale=.92]{roboto}
\renewcommand{\familydefault}{\sfdefault}


\title{Writing presentations in \LaTeX\ beamer?}
\subtitle{Does this make sense?}
\author[Sebastian Friedl]{Sebastian Friedl (\href{mailto:sfr682k@t-online.de}{\texttt{sfr682k@t-online.de}})}
\institute{\itshape Dedicated to both, \LaTeX\ beamer's developers and (possible) users}

\begin{document}
	\frame{\maketitle}
	
	
	\begin{frame}{Contents}
		\tableofcontents
	\end{frame}
	
	
	\section[Why \LaTeX\ beamer?]{Why should I use \LaTeX\ beamer?}
	\begin{frame}{Why should I use \LaTeX\ beamer?}{A simple argument}
		\textbf{\color{red} Because it's better!} Ever done stuff like \textbf{this} \dots
		
		
		\vfill
		\begin{block}{Coulomb's law}
			It is commonly known that the force $\color{red} F$ interacting between statical electrical particles is described by
			\begin{equation*}
				{\color{red} F} = {\color{green!40!black} k_\varepsilon} \frac{\color{blue} q_1 q_2}{{\color{red!50!blue} r}^2}
			\end{equation*}
			where $\color{blue} q_1$, $\color{blue} q_2$ represent the signed magnitude of the charges and $\color{red!50!blue} r$ the distance between them.  $\color{green!40!black} k_\varepsilon$ is defined as $\frac{1}{4\pi \varepsilon_0}$.
		\end{block}
		\vfill
		
		
		\dots in PowerPoint\footnote{or one of its Open Source alternatives?}? \textbf{IMPOSSIBLE \dots}
	\end{frame}

	
	\section{Advantages}
	\begin{frame}[fragile]{Advantages}
		\begin{itemize}
			\item You can use normal \LaTeX\ code
			\begin{itemize}
				\item Structure your presentation with \verb|\section| \& Co. and generate a table of contents with \verb|\tableofcontents|
				%
				\item Insert formulas like you are used to
				%
				\item Useful, presentation specific commands available:
				\begin{itemize}
					\item Create overlays (e.~g.~with \verb|\uncover| or \verb|\only|)
					\item Color blocks and boxes
				\end{itemize}
			\end{itemize}
			%
			\item Your presentation always looks like the same \dots \\
			\textbf{on every computer and platform!!} \\
			Only a program capable of displaying \texttt{.pdf} files in fullscreen mode is required
		\end{itemize}
	\end{frame}


	\section{Disadvantages}
	\begin{frame}{Disadvantages}
		\begin{enumerate}
			\item \textbf{Big disadvantages:}
			\textit{There are none!}
			
			\item \textbf{\enquote{Real} disadvantages:}
			\begin{enumerate}
				\item Some things easily possible in PowerPoint are hard to fulfil and need some \enquote{out of the box}--thinking
				\item The ugly Computer Modern fonts are used by default
			\end{enumerate}
			
			\item \textbf{No real disadvantages:}
			\begin{enumerate}
				\item You can simply mix itemizations and enumerations:
				\begin{itemize}
					\item You're getting confused about that?
					\item It's your problem, not mine \dots
				\end{itemize}
				
				\item You'll have to learn some \LaTeX\ commands if you didn't do so
				\begin{enumerate}
					\item \LaTeX\ is an enrichment and capable of (nearly) everything!
					\item \enquote{Out of the box}--thinking trains your brain!
				\end{enumerate}				
			\end{enumerate}
		\end{enumerate}
	\end{frame}
	
	\section{Conclusion}
	\begin{frame}{Conclusion}
		\begin{block}{Switch to \LaTeX!}
			\LaTeX\ is nearly almighty! You can even typeset sudokus with it \dots
		\end{block}
	
		\begin{exampleblock}{\LaTeX\ doesn't cost a single cent!}
			Just download the current version of \TeX Live from \href{http://tug.org}{\texttt{tug.org}}
		\end{exampleblock}
	
		\begin{alertblock}{\textbf{Warning!}}
			Actually, some people are simply too dumb for using \LaTeX\ or recognizing the big advantages of it.
		\end{alertblock}
	\end{frame}
	
	
	\appendix
	
	% COMMENT THIS FRAME IF THE THEME AUTOMATICALLY PROVIDES A SEPARATION FRAME AT BEGINNING OF THE APPENDIX
%	\frame[c]{\centering\bfseries\Huge \appendixname}
	
	\begin{frame}[fragile]{Appendix frames}
		For creating appendix frames, just begin an appendix part with the \verb|\appendix| command like you do in other \LaTeX\ documents.
		
		\medskip
		If you don't want appendix frames being counted into the total number of frames you may load the \texttt{appendixnumberbeamer} package in your document's preamble.
	\end{frame}
\end{document}
